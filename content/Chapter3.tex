\chapter{不收敛}
\label{chap:non-conver}

\section{理解不收敛}
很多SPICE用户曾经一次或者其他什么时候观察到仿真会收敛失败。不收敛是非线性解算法的一种失效;它意味着仿真器无法找到一组满足Kirchoff电压和电流定律的节点电压和支路电流。不收敛或许是一个持续时间最久,让仿真用户最头疼的问题。

但,是什么引起了不收敛?是因为解算法还是因为执行的仿真类型?模型在不收敛中扮演角色吗?在这些中最重要的问题是,有什么办法可以被用来减少或者消除不收敛?

尽管计算机仿真伴随我们已经超过20年,而且从计算机仿真诞生开始不收敛就一直跟着我们,但是几乎没有什么关于不收敛的原因和诊治的办法被写下来。很多有经验的仿真用户学会克服不收敛问题是依靠多年从试验和错误的实验中得到的经验。但是通过试验和错误来学习是低效的、低产的和非常令人沮丧的。

在本章中,引起不收敛的机制会被探寻和解释。本章中的每一节会检查一种不同的不收敛机制。一旦不收敛机制清楚了,一种简单的、有条理的、系统的技术会被用来展示消除不收敛。

\section{收敛性和Newton-Raphson算法}
关于不收敛问题的学习最好从SPICE如何在一系列节点电压上收敛的简要讨论开始。理解了SPICE在一系列节点电压上如何收敛也就能比较好地理解SPICE为什么有时会收敛失败。第\ref{chap:2}论证了Newton-Raphson算法是如何应用到一系列非线性电路方程的,但是通过图示的方法看Newton-Raphson过程还能学到很多。

为了说明Newton-Raphson迭代求解技术,图\ref{图3.1}展示了第\ref{chap:2}章中用过的简单二极管电路。这个同样的电路也会被用来图示说明Newton-Raphson过程。
\begin{figure}[htbp]
\small
    \centering
    \includegraphics[width=0.5\textwidth]{figure/Chapter3/图3.1.png}
    \caption{一个简单的二极管电路。}
    \label{图3.1}
\end{figure}

图\ref{图3.2}(a)展示了二极管电路的I-V特性曲线,\ref{图3.2}(b)展示了电阻和电流源影响下的负载线。如果图\ref{图3.2}(a)和(b)是可叠加的,在节点VD处的电压或许可以由图\ref{图3.3}中的二极管I-V特性曲线和负载线的交点确定。这是一种寻找节点VD处电压的图示技术。
\begin{figure}[htbp]
\small
    \centering
    \includegraphics[width=0.5\textwidth]{figure/Chapter3/图3.2.png}
    \caption{图\ref{图3.1}二极管电路的I-V特性曲线(a);图\ref{图3.1}线性元件的负载线。}
    \label{图3.2}
\end{figure}
\begin{figure}[htbp]
\small
    \centering
    \includegraphics[width=0.5\textwidth]{figure/Chapter3/图3.3.png}
    \caption{图\ref{图3.1}电路的解可以通过负载线和二极管的I-V特性曲线的交点得到。}
    \label{图3.3}
\end{figure}

在SPICE中,VD电压通过Newton-Raphson算法找到。图\ref{图3.4ab}(a)展示了二极管I-V特性曲线和二极管电路负载线的超位置。Newton-Raphson算法从一个初始电压值开始搜索两条曲线的交点。当初值($V_0$)确定后,SPICE会把$V_0$电压传递给LOAD子函数。LOAD子函数决定电压$V_0$处的二极管线性模型。图\ref{图3.4ab}(b)展示了重叠在图\ref{图3.4ab}(a)中的二极管线性模型。一旦线性模型确定,Newton-Raphson算法就会通过定位线性模型和电路负载线的交点来计算第二次迭代的电压值($V_1$),如图\ref{图3.4cd}(c)所示。
\begin{figure}[htbp]
\small
    \centering
    \includegraphics[width=0.7\textwidth]{figure/Chapter3/图3.4ab.png}
    \caption{(a)二极管I-V特性曲线和负载线的超位置以及迭代的初值$V_0$;(b)在$V_0$处二极管特性曲线的线性近似。}
    \label{图3.4ab}
\end{figure}

随着电压$V_1$处的二极管线性模型被重新计算,迭代过程将继续下去。新的线性模型会被再次延展到电路负载线,以此来预测第三次迭代电压值($V_2$),如图\ref{图3.4cd}所示。
\begin{figure}[htbp]
\small
    \centering
    \includegraphics[width=0.7\textwidth]{figure/Chapter3/图3.4cd.png}
    \caption{(c)二极管特性的线性近似和负载线决定迭代电压$V_1$;(d)二极管曲线在$V_1$处的线性近似和负载线决定$V_2$。}
    \label{图3.4cd}
\end{figure}

Newton-Raphson迭代会一直进行,最终的迭代电压值将趋近二极管电流和电路负载线的交点。当迭代电压值趋近两条曲线的精确交点的时候,连续电压迭代之间的差会逐步减小,如图\ref{图3.5}所示。SPICE在解处收敛,当电压和电压迭代进入到一个预定义的误差容限内时会停止迭代。图\ref{图3.6}展示了迭代节点电压值和迭代之间的误差。误差容限是解算法中一个重要的部分,决定了SPICE在终止迭代进程前收敛到精确解的接近程度。用户可以在.OPTIONS声明中定义误差容限。
\begin{figure}[htbp]
\small
    \centering
    \includegraphics[width=0.7\textwidth]{figure/Chapter3/图3.5.png}
    \caption{二极管电路在初值估计为1.5volts的情况下不同迭代下的电压值。(复印自Successfully Simulating Circuits with SPICE。授权使用。)}
    \label{图3.5}
\end{figure}

\begin{figure}[htbp]
\small
    \centering
    \includegraphics[width=0.7\textwidth]{figure/Chapter3/图3.6.png}
    \caption{二极管电路中不同迭代下的电压值和迭代之间的误差。(复印自Successfully Simulating Circuits with SPICE。授权使用。)}
    \label{图3.6}
\end{figure}

\section{不收敛的起因}
仿真器收敛失败的原因有几个。其中一些可能来自Newton-Raphson算法,一些与被执行的分析类型有关,其他的与器件模型有关。因为有几种不收敛的机制,在克服不收敛失效中学会识别出不收敛的起因是一门重要的学问。
\section{克服不收敛性失败}
有经验的用户会主动测试不收敛问题,而新用户或许只会偶尔看到问题。对于那些没有见过不收敛现象的用户,用如下命令仿真电路文件ch3-1.cir:

SIM CH3-1.CIR

然后查看输出文件。在这个仿真中,SPICE无法为电路确定合适的DC偏置。

ch3-1.cir是一种典型的不收敛失效,但是不收敛问题可以通过.OPTIONS声明参数的合理使用来纠正。实际上,某种程度上75\%~90\%的不收敛发生率可以通过对.OPTIONS声明中的参数或者.MODEL声明中的模型参数的合理使用来消除。

不幸的是,对于很多用户来说,.OPTIONS声明参数从来没有被清楚地定义过,或者参数真正的意思没有被解释过。因为缺乏好的文档资料,很多用户简单地依靠尝试不同的选项设置,直到其中的某个可以纠正不收敛问题。但是这种技术既不非常科学,又不非常可靠,往往只是碰巧成功。

在本章,不收敛的起因会被解释,并且对于每一种起因,纠正问题起因的简单的循序渐进的步骤会被支持。本章集中在解决不收敛问题。

\subsection{解决办法}
在解决不收敛问题中,几个不同的起因必须被测试。一些不收敛问题是因为在SPCIE中使用Newton-Raphson算法引起的。这些问题会在名为“广义Newton-Raphson收敛辅助”小节解决,而且这里提到的技术会被应用到SPICE中的每一种非线性分析。广义Newton-Raphson收敛辅助在DC工作点分析,DC扫描分析和瞬时扫描中会减少不收敛问题。广义Newton-Raphson收敛辅助给收敛定义误差,并且给电路设置最小和最大容许的电导值。很多不收敛问题通过设置广义Newton-Raphson收敛辅助的参数会被消除。因为收敛辅助适用于所有分析类型,所以广义Newton-Raphson收敛辅助在分析开始前应该被应用到每一种仿真上。

但是广义Newton-Raphson收敛辅助不会消除掉设计人员看到的所有不收敛问题。很多不收敛失效可以与被执行的分析类型联系上。因为这个原因,名为“不收敛和DC工作点解”,“不收敛和DC扫描分析”以及“不收敛和瞬态分析”的小节会查看与特定分析类型相关的不收敛根源。在这些小节中,不收敛的起因同样会被讨论,而且会给出简单的循序渐进的步骤来消除不收敛的起因。

本质上讲,这章是解决不收敛问题的一个食谱形方法。第一,设置广义Newton-Raphson收敛辅助参数。之后,如果你的仿真因为不收敛而失败,那么确定不收敛发生时在执行哪种SPICE分析类型后遵循该分析类型的步骤即可。

\section{广义Newton-Raphson收敛辅助}
广义Newton-Raphson收敛辅助是一系列.OPTIONS声明参数和模型参数,这些参数应该根据你的电路设置到合适的水平。广义Newton-Raphson收敛辅助定义精度。SPICE必须在一个解上收敛,然后停止迭代过程。广义Newton-Raphson收敛辅助也决定最小和最大容许的电路电阻值;这些值会决定SPICE能有多快收敛到合适的解。SPICE的缩写代表着Simulation Program with Integrated Circuit Emphasis。在1960年代晚期到1970年代早期,面包板被发现对很多集成电路是几乎不可能的,另外分析集成电路行为的需求对成功的设计至关重要。很多SPICE开发的赞助来自集成电路工业界,所以SPICE中很多默认设置的值是为这些类型的电路选择的。

但是今天,除了集成电路,SPICE还被用在仿真离散的,板级的,和甚至高压电路。很多程序的默认设置对这种类型的电路不再适用。广义Newton-Raphson收敛辅助重置程序的默认设置为对被仿电路合适的水平。

广义Newton-Raphson收敛辅助包含学习设置精度误差容限(RELTOL, VNTOL,和ABSTOL),学习设置最小电导GMIN,和学习设置半导体模型的系列电阻参数。

\subsection{精度误差容限}
在CANCER和SPICE1的早期版本中,当程序找到一系列满足Kirchhoff电压定律或者足够接近这些电压的节点电压时,就认为实现了收敛。对于二极管,双极型晶体管和其他有着指数型I-V特性的器件来说,电压上的一个小的变动会在器件电流上产生一个大的变动。这种行为如图\ref{图3.7}所示。不论是CANCER还是SPICE1,都不能检查出非线性器件支路电流的收敛性。经常,仿真带指数型I-V特性的电路得到的结果不满足Kirchhoff电流定律;正因为此,SPICE的后续版本(包括SPICE2的所有版本)除了检查电压收敛的机制外还增加了电流收敛的检查机制。这意味着不仅需要电压到电压迭代必须收敛到解,非线性器件支路电流也必须收敛到解。这也意味着电压和电流误差容限都必须被定义。

\begin{figure}[htbp]
\small
    \centering
    \includegraphics[width=0.5\textwidth]{figure/Chapter3/图3.7.png}
    \caption{对于前向偏置的PN结,电压上的小变动会导致电流上的大变动。}
    \label{图3.7}
\end{figure}

误差容许因为两个原因是重要的。第一,误差容限定义了解的精度。第二,误差容限定义了找到解需要进行多少次迭代。图\ref{图3.8}展示了为了达到给定水平的精度,不同误差容限会要求更多或者更少的迭代次数。
\begin{figure}[htbp]
\small
    \centering
    \includegraphics[width=0.5\textwidth]{figure/Chapter3/图3.8.png}
    \caption{为了得到较高水平的解精度,需要更多的迭代次数。(重印自Successfully Simulating Circuits with SPICE。授权使用)}
    \label{图3.8}
\end{figure}

在SPICE中,电压误差容限和电流误差容限都是由一个相对限制和一个绝对限制构成的。图\ref{图3.9}展示了SPICE在迭代过程中用来检查收敛解电压的流程。
\begin{figure}[htbp]
\small
    \centering
    \includegraphics[width=0.7\textwidth]{figure/Chapter3/图3.9.png}
    \caption{检查收敛解电压。}
    \label{图3.9}
\end{figure}

在迭代过程中,当SPICE接近电路方程的精确节点电压集合时,每一步迭代电压变化和每一步迭代电流变化会变得非常小。理想情况下,SPICE会在每一步迭代电压变化和每一步迭代电流变化等于0时停止迭代。在这种情况下,迭代之间的电压和电流值会完全相同。但是因为数字计算机舍入误差的存在,当所有节点的每一步节点电压和所有非线性支路的每一步电流小于图\ref{图3.9}中定义的误差容限时,SPICE就会定义收敛。

在每一次Newton-Raphson迭代完成后,SPICE会检查前一步和当前步的迭代节点电压之间的误差,当然也会检查前一步和当前步迭代支路电流之间的误差。一旦SPICE发现一个在两次或者多次迭代间比较接近的一致的解,迭代就会停止,解电压和电流也会保存。如果在容许的迭代次数内找不到解,SPICE会终止迭代过程,打印不收敛警告信息,并暂停仿真。

\subsubsection{相对误差容限}
误差容限是通过.OPTIONS声明参数RELTOL, VNTOL和ABSTOL定义的。RELTOL定义了收敛的相对误差容限。相对误差容限要求迭代过程一直进行到每一步迭代电压变化和每一步迭代电流变化小于最终结果的一个百分比。例如,RELTOL默认设置为.001 (百分之.1)。对于一个5-volt的电路节点,为了满足相对误差容限收敛准则,在迭代之间预测的节点电压变化必须是5mV或者更小。当节点电压在前一次迭代电压的5mV变化范围内时,SPICE会停止迭代过程。为了终止迭代过程,5mA的支路电流必须收敛到5$\mu$A或者更小的值。用户可以通过.OPTIONS声明重置RELTOL。

\subsubsection{绝对误差容限}
在图\ref{图3.9}中,相对误差容限RELTOL通过绝对误差容限VNTOL和ABSTOL来补充。绝对误差容限是必不可少的,因为当一个节点电压或者支路趋近或穿过0时,相对误差容限也会趋于0,这意味着仿真结果一定是无限精确的。为了避免这些条件下的问题,SPICE的作者给收敛检查中加入了绝对误差容限VNTOL和ABSTOL。

如图\ref{图3.9}所示,为了达到收敛,每步迭代电压变化和每步迭代电流变化必须小于相对误差容限和绝对误差容限的和。但是VNTOL有一个仅有1$\mu$V的默认值,而ABSTOL有一个仅有1pA的默认值。在正常的工作条件下,相对收敛误差比绝对收敛误差大的多,所以在这些条件下,收敛是由相对收敛误差决定的。但是当节点电压或者支路电流等于或者趋近于0时,相对误差容限也会等于0。在这些条件下,绝对误差容限决定什么时候迭代收敛。

\subsubsection{设置误差容限}
对于很多类型的电路,RELTOL的默认设置都可以产生一个可接受的精度和一个合理的仿真运行时间。很少的一些电路会要求更精确的解(更小的RELTOL),而一些也许会要求精度更低的解(更大的RELTOL)。RELTOL的值可以在.OPTIONS声明中设置如下。

.OPTIONS RELTOL=.0001

改变RELTOL的值会改变仿真需要的时间。较小的RELTOL可以增加结果的精度但是需要更长的仿真时间,因为需要额外的迭代来实现额外的精度。较大的RELTOL降低结果的精度但是仿真会更快。作为一种通用的指导原则,RELTOL每降低10倍(更精确的结果)大概会使得求解电路的迭代次数翻番。RELTOL每增加10倍(更低精度的结果)大概会使得求解电路的迭代次数减半。

RELTOL,VNTOL和ABSTOL的默认值设置是为了在1970年代晚期针对那时的电路产生可以接受的结果。(记着SPICE是Simulation Program with Integrated Circuit Emphasis的缩写。)但是今天,SPICE被用来仿真很多不同类型的电路,包括离散的,板级的和高压电压。虽然不管电路中的电压和电流水平RELTOL仍然可以产生出相同百倍比的精度,但那些包含的电压和电流水平比典型集成电路中的高很多或者低很多的电路就不得不重置VNTOL和ABSTOL到适当的水平。(你真的需要把你的20-volt开关电源仿真到1$\mu$V和1pA的精度吗?)

广义Newton-Raphson收敛的第一个辅助就是合理选择仿真的误差容限。设置误差容限可以遵循图\ref{图3.10}中的步骤。虽然设置误差容限需要电路电压和电流水平的知识,但把误差容限设置到适合你电路的水平会带来更快的仿真速度和更少的不收敛失效。
\begin{figure}[htbp]
\small
    \centering
    \includegraphics[width=0.7\textwidth]{figure/Chapter3/图3.10.png}
    \caption{设置SPICE误差容限。}
    \label{图3.10}
\end{figure}

\subsection{电路电导值}
像误差容限一样,电路的电导值也决定着SPICE能够多快收敛到解。但是几乎没有SPICE用户意识到电导项的重要性。当SPICE开始迭代的时候,电压和电流的解必须在容许的迭代次数内找到解,而且尽管最大和最小的电导值对电路的精度影响很小,但电导的上界和下界会影响到Newton-Raphson算法能多快收敛到解。
\begin{equation}
    V_{n+1}=V_n - \frac{F(V_n)}{G(V_n)}
    \label{eq:3.1}
\end{equation}

方程(\ref{eq:3.1})是Newton-Raphson算法应用到一个单节点电路时的方程。在方程(\ref{eq:3.1})中,第二项的分母就是电路元件的电导值。Newton-Raphson算法用电导值来预测下一次电压迭代会落在哪里。如果电导非常小,方程(\ref{eq:3.1})中的第二项就会非常大。更糟的是,如果电导值趋近于0,下次Newton迭代就会引起浮点除以0的错误,导致程序崩溃。

图\ref{图3.11}展示了反向偏置条件下二极管特性。在这个区域,SPICE把I-V特性建模为一种恒定电流-IS。当电流恒定,且不再是二极管电压的函数时,二极管电导就变为了0。在这些条件下,Newton-Raphson算法会失败因为二极管电导为0。

\begin{figure}[htbp]
\small
    \centering
    \includegraphics[width=0.5\textwidth]{figure/Chapter3/图3.11.png}
    \caption{二极管电导($g_d$)在反向偏置工作区会落到0上。}
    \label{图3.11}
\end{figure}

因为所有的半导体器件都包含一个或者多个零电导区(恒定电流输出),电导问题必须被解决。SPICE的作者们通过给SPICE中的每一个半导体模型中的每一个PN结并联一个分流电阻来解决这个问题。该电阻的默认值为1000G ohms,且该电阻的值可以用.OPTION GMIN=X来设定。GMIN代表该分流电阻的电导(电阻的倒数)。该GMIN电阻被加入到模型方程中,经常给器件提供一段小的压控电流。方程(\ref{eq:3.2})是带GMIN电阻项的在反向偏置下的二极管,而方程(\ref{eq:3.3})是二极管的电导。

\begin{equation}
    I_d = -IS + V_d *GMIN
    \label{eq:3.2}
\end{equation}

\begin{equation}
    G_d = \frac{dI_d}{dV_d}=0 + GMIN
    \label{eq:3.3}
\end{equation}

GMIN电阻存在于二极管,双极型晶体管的基极-发射极和基极-集电极,JFET的栅极-漏极和栅极-源极结,以及MOSFET的漏极-体极和源极-体极结中。SPICE中的每个器件的每个PN结都包含GMIN电阻。正常情况下,电阻值非常高以使电流在仿真误差容限内。这意味着流过GMIN电阻的电流不会对仿真结果中的值造成影响。例如,如果一个电路二极管保持在-5V的反向偏置和1$\mu$A的反向饱和电流下,流过GMIN电阻的电流将是5pA(5V/1000G ohms=5pA)。5pA比SPICE的1nA相对误差容限(RELTOL)要小(1$\mu$A * .001 = 1nA)。GMIN的默认值是根据仿真精度不会被其存在影响的前提下选择的。

但是GMIN是如何影响电路的收敛特性的?只要方程(\ref{eq:3.1})中的电导项是有限的,Newton-Raphson算法就会继续收敛到一个解,但是如果电导非常小,方程(\ref{eq:3.1})中的第二项(F($V_n$)/($V_n$))就变得非常大,下次迭代的电压值会离上次的电压值很远,如图\ref{图3.12}所示。经常,非常小的电导值会强使Newton-Raphson算法越过正确的解电压。当这种情况发生时,Newton-Raphson算法会需要几十次迭代来回到正确的解电压上。

这个效果可用通过在文本软盘下的ch2子目录下的DIODE.EXE程序的使用验证。从.5 volts的初始二极管电压开始,第一次迭代是9.6 volts!为了达到.7291的最终解电压,需要额外360次迭代。如果初始二极管电压是.55 volts,第一次迭代是7.8 volts,那么到达.7291 volts需要289次额外的迭代。.6 volts的初始电压会在第一次迭代时产生3.6 volts的结果,在到达.7291 volts时需要118次额外的迭代。

SPICE必须在固定迭代次数内找到合适的解。因为这个限制,用尽可能少的迭代找到解是实现收敛的一个重要方面。对仿真用户来说,GMIN的值越大,Newton-Raphson算法就会越快收敛到解。为了帮助避免不收敛问题,在不影响仿真输出精度的前提下GMIN应该被尽可能地往大设置。提升GMIN会降低分流电阻的尺寸。只要分流电阻的电流影响小于电流相对误差容限的精度,在仿真结果的精度方面就不会有什么差别。

\begin{figure}[htbp]
\small
    \centering
    \includegraphics[width=0.5\textwidth]{figure/Chapter3/图3.12.png}
    \caption{小电导值(大电阻值)会导致迭代之间的大的电压变化。}
    \label{图3.12}
\end{figure}

图\ref{图3.13}展示了如何针对你的电路选择合适的GMIN值。
\begin{figure}[htbp]
\small
    \centering
    \includegraphics[width=0.7\textwidth]{figure/Chapter3/图3.13.png}
    \caption{设置GMIN值的步骤。}
    \label{图3.13}
\end{figure}

GMIN是对整个电路的一个全局设置,所以当选择GMIN的值时,应该考虑电路中对电流最敏感的部分。

就像在遇到非常小的电导值时会出现问题一样,遇到非常大的电导值时Newton-Raphson算法也会出现问题。图\ref{图3.14}展示了在高前向偏执条件下一个二极管的I-V特性曲线。对于非常大的电导值,方程(\ref{eq:3.1})中第二项变得非常小。这样的话,下一次电压迭代($V_{n+1}$)和$V_n$相比只会有细微的不同,并且如果器件离合适的解电压偏置远的话,需要很多次迭代才能回到合适的解上。

\begin{figure}[htbp]
\small
    \centering
    \includegraphics[width=0.5\textwidth]{figure/Chapter3/图3.14.png}
    \caption{大的电导值(小的电阻值)引起迭代之间小的电压变化。}
    \label{图3.14}
\end{figure}

当二极管或者PN结的前向偏置超过.8 volts时,高电导值是非常麻烦的。超过这个点后,电导值会变得不切实际地大(电阻变得非常小),进而引发不收敛问题。虽然几乎没有电路会要求二极管偏置如此之高,但是在迭代过程中,在迭代到解时,SPICE也许会给电路中的任意一个二极管施加一个高的前向偏置。如果这种情况发生,SPICE在到达合适的电路解之前会无法收敛。

但是SPICE用户通常可以通过给所有的电路二极管和双极型器件专门指定系列电阻模型参数来守护仿真。(虽然经常给MOSFET和JFET器件设置系列电阻是个好习惯,但几乎没有什么意愿来给器件的内部PN结设置前向偏置。)系列电阻的默认值是0 ohms,而如果在迭代过程中SPICE给一个二极管或者PN结施加了高的前向偏置电压,那么不收敛的条件也许就会发生。但是如果系列电阻项不为0,在高前向偏执条件下,系列电阻会决定器件的电导,并且帮助减少不收敛情况的发生。

在每一个半导体器件的.MODEL参数声明中都可以找到系列电阻项。系列电阻项并不是.OPTION声明中的全局参数。检查一下你的电路的模型参数。一个好的模型应该总是定义了系列电阻。如果模型在参数列表中没有电阻项,为系列电阻选择一个足够小的以保证对电路的操作没有明显影响的值,并把它加到模型参数中。表\ref{表3.1}展示了每一种半导体器件的系列电阻项。

\begin{figure}[htbp]
\small
    \centering
    \includegraphics[width=0.7\textwidth]{figure/Chapter3/表3.1.png}
    \caption{系列电阻模型参数}
    \label{表3.1}
\end{figure}

\section{广义Newton-Raphson收敛辅助总结}
广义Newton-Raphson收敛辅助包括学习给电路设置误差容限,选择反映了电路中最大电阻的GMIN的值,和给二极管以及双极型晶体管设置系列电阻为非零值。通过遵循这些指导原则,仿真会运行得更快并且更少有机会收敛失效。表\ref{表3.2}展示了广义Newton-Raphson收敛辅助。
\begin{figure}[htbp]
\small
    \centering
    \includegraphics[width=0.7\textwidth]{figure/Chapter3/表3.2.png}
    \caption{广义Newton-Raphson收敛辅助}
    \label{表3.2}
\end{figure}

\section{分析专用的收敛性辅助}
指定广义Newton-Raphson收敛辅助会减少不收敛的发生率。但是很多收敛失败与一种或者多种分析类型紧密关联。在很多这样的例子中,仅靠广义Newton-Raphson收敛辅助是不足以消除非收敛的。在大多数特定分析的收敛失效中,通过理解是什么造成了问题和设置合适的.OPTION声明会纠正不收敛。克服特定分析的不收敛失效的关键是理解什么造成了失效,但是理解不收敛的起因偏偏是大多数工程师有问题的地方。

SPICE有能力处理很多不同的分析类型,但只有其中三种可能会发生不收敛失效。SPCIE可能会在直流工作点计算,直流扫描分析或者瞬时扫描中收敛失败。在克服特定分析不收敛中,对于一个给定的分析SPICE遵循的事件顺序必须搞清楚。表\ref{表3.3}展示了四种主要的SPICE分析类型的分析流程。

\begin{figure}[htbp]
\small
    \centering
    \includegraphics[width=0.7\textwidth]{figure/Chapter3/表3.3.png}
    \caption{SPICE中的分析流程}
    \label{表3.3}
\end{figure}

SPICE中的每一种分析类型都从直流偏置点的计算开始。出于这个原因,相比于其他特定分析的不收敛辅助,SPICE的作者们加了更多的直流偏置点不收敛辅助。

\section{直流偏置点收敛性辅助}
确定电路直流偏置点的工作既是最困难的也是最重要的。直流偏置点的重要性体现在它作为所有其他分析类型的初始电路状态。直流偏置点计算的困难性体现在SPICE经常有很少或者压根没有关于电路应该如何偏置的信息。(和直流扫描分析或瞬时扫描比较,前一次的解电压被设置为下一系列Newton迭代的初值。)

功能上讲,DC偏置点的计算早在本章和第\ref{chap:2}章开头的二极管电路例子中就讲过了。但是具体的关于DC偏置点的计算是如何进行以及.NODESET声明和ITL1选项的影响还没有被展示。

\subsection{直流偏置点计算}
如第\ref{chap:2}章所示,SPICE从输入文件的元件来构造系统(矩阵)方程。一旦系统方程准备到位,SPICE需要给电路节点电压一个初始猜测以启动Newton-Raphson算法。

SPICE通过给所有连接电压源的节点设置电压源0时刻或者直流电压水平的值来完成初始节点电压阵列的猜测。在电流阵列中代表与电路电流源连接的节点会被设置合适的电流源0时刻或者直流电流水平的值。电路中的所有剩余节点会被设置为0.这就形成了启动Newton-RaphsonS算法的初始猜测。

一旦初始猜测输入到电压阵列中,SPICE就开始迭代进程。Newton迭代会一直持续下去直到所有的节点和所有的非线性支路电流在特定的误差容限内收敛或者迭代次数超过ITL1。ITL1是.OPTIONS声明参数,决定了SPICE为了确定电路的直流工作点最多可以进行多少次迭代。如果所有的节点和支路电流在ITL1次迭代用完之后还没有收敛到特定的误差容限内,SPICE会终止迭代进程并且打印出可怕的“在DC工作点不收敛”的错误信息。在SPICE中ITL1默认值为100。

如果SPICE在直流偏置点收敛失败,原因经常会追溯到要么缺乏一个启动Newton-Raphson迭代的好初值,要么对于一个特定的错误容限缺乏足够的迭代次数以达到一个满意的解。

\subsection{提升ITL1}
第一个直流偏置点不收敛辅助是ITL1。很多不收敛的情况可以通过简单地在.OPTIONS声明中提升ITL1来消除。ITL1的默认设置是在1970年代中期,因为那时仿真多于几十个节点时会比较慢,大家通常会利用这段时间去休息喝咖啡。今天有几百甚至几千个节点的电路很常见。许多电路为了达到一个稳定的偏置点会需要多于100次的迭代。直流偏置点不收敛辅助的第一条措施就是提升ITL1。ITL1设置了直流偏置点搜索终止前的SPICE用的迭代次数的限制。通过提升ITL1,如果需要,SPICE会进行额外的迭代。提升ITL1的.OPTIONS声明如下所示。

.OPTIONS ITL1 = 500

表\ref{表3.4}展示了迭代次数和无其他收敛辅助措施的情况下电路收敛到稳定的直流偏置点的百分比。

从表\ref{表3.4}可以看出,设置ITL1为500能产生最好的结果。在测试过的电路中,设置更高的ITL1并不能看到可测试提升。对于所有的电路,在.OPTIONS声明中设置ITL1为500。一些电路或许需要更多,或更少的迭代。ITL1是第一个设置的直流工作点不收敛辅助措施。

\begin{figure}[htbp]
\small
    \centering
    \includegraphics[width=0.7\textwidth]{figure/Chapter3/表3.4.png}
    \caption{SPICE中的迭代次数}
    \label{表3.4}
\end{figure}

作为一个例子,用如下命令仿真ch3-15a.cir电路文件:

SIM CH3-15A.CIR

100次迭代后,仿真会终止并警告“直流工作点不收敛”。然后仿真ch3-15b.cir。除了.OPTIONS声明中ITL1=500外,ch3-15b电路和ch3-15a电路完全相同。ch3-15b电路容许SPICE用要求之外的额外的迭代来找到正确的电路偏置。

\subsection{设置初始节点电压}
第\ref{chap:2}章的DIODE.EXE程序展示了有一个靠近最终值的初始猜测对启动Newton-Raphson算法的重要性。设定初始节点电压可以减少抵达解需要的Newton-Raphson迭代次数。在SPICE中,当用.NODESET声明启动分析时可以给电路节点设置初值。下面是用.NODESET声明来设置几个电路节点电压的例子。

.NODESET V(2)=12.5 V(5)=5.0 V(7)=16.8

.NODESET声明通过给特定节点施加Norton等效电压源来强制给特定节点设置专门的电压。Norton等效源如图\ref{图3.15}所示。电源包含了一个1-ohm的电阻和一个电流源。电流源的电流值设置为该节点在.NODESET声明中被指定的电压。对于大多数电路,在特定节点看到的负载远远大于1 ohm。在这些条件下,来自生成器的所有电流都被强迫流过电阻,以强使节点改变到合适的电压。

\begin{figure}[htbp]
\small
    \centering
    \includegraphics[width=0.5\textwidth]{figure/Chapter3/图3.15.png}
    \caption{为了模拟.NODESET电压在SPICE中用到的Norton等效电压源。}
    \label{图3.15}
\end{figure}

当SPICE遇到一个或者多个.NODESET声明时,仿真器会进行两次直流偏置点计算。第一次计算是在Norton等效源保持在电路中时完成。如果带着准备好的Norton等效源的电路收敛了,偏置点电压就会被保存,Norton等效源会被移除,第二系列的Newton-Raphson迭代将决定最终的偏置解。第二次偏置点计算是为了确保Norton等效源不会在电路上加上一个负载。

有单个稳定偏置点的中等复杂度的电路很少要求使用.NODESET声明。通常,如果非常大的电路中的几个电路节点用.NODESET设置了特定的值,那么该电路收敛得会更快些。但是.NODESET声明最好用在不止有一个稳定工作点的电路中。例如,触发器和锁存器会在高或者低的状态下开始工作。对于这种类型的电路,.NODESET声明可以被用来提高收敛率并确保电路在合适的状态下启动。

通常在直流偏置点不收敛的例子中,不收敛节点的数目比较少,一般一个或者两个。当SPICE中止直流偏置点的计算时,最后一次迭代的节点电压值会被打印到输出文件中。检查这些节点值通常会发现只有一个或者两个节点的值偏离了合适的偏置值。这些就是引起不收敛问题的节点,用.NODESET声明设置这些节点经常可以在下一次仿真中成功定位到偏置点。

关于.NODESET声明的一句提示:不要把.NODESET声明用到没有稳定直流偏置点的电路中(比如振荡器和其他不稳定电路)。当.NODEST声明被用到不稳定电路中时,尽管Norton等效源被附加到了电路上,SPICE会收敛,但是在第二波迭代中,当源被移除,电路的不稳定性将无法定位到稳定的已存在的条件,进而会引发仿真器收敛失败。SPICE确实有启动不稳定电路的机制,.IC声明。但是关于.IC声明的讨论会被推迟到瞬态分析小节之后。

\subsection{电源步进}
在大多数情况下,直流工作点不收敛可以通过增加ITL1和合理使用.NODESET声明来消除。但是有些电路证明即使对这些措施也抵抗,而且有些电路含有子电路。子电路中的节点不能用.NODESET声明来设置。对于这种类型的电路,电源步进算法会被用来计算直流偏置。

电源步进是一种使电路电源从0(此时所有节点的解都是0)步进到满值的技术。在每一步,前一次节点电压会被用来设置为下一波Newton-Raphson迭代的初值。

当.OPTION ITL6=X被设置为一个非零值时,电源步进算法会被用来计算直流偏置点。ITL6是一个在进程中决定每步被允许的迭代次数的参数。ITL6是针对电源步进的,就像ITL1是针对正常的直流偏置点的计算。对于提高ITL1和/或增加.NODESET声明后还不能收敛的电路,可以在.OPTIONS声明中设置ITL6为500或者更多。当ITL6被设置一个非零0值,电源步进算法会取代正常的直流偏置点计算。

为了证明电源步进算法,用以下的命令仿真ch3-18a.cir硬盘文件:

SIM CH3-18A.CIR

这个大电路会在直流偏置点处收敛失败。由于电路的规模和复杂性,合适的.NODESET值不能很快被确定。通过用.OPTIONS ITL6=500取代.OPTIONS ITL1=500声明,在电源步进算法下电路偏置点会被迅速定位。用如下的命令仿真纠正的电路:

SIM CH3-18B.CIR

在这里,对于所有的电路,读者可能会被诱导去用电路步进算法来替代正常的直流偏置点算法。虽然电源步进算法在很多不收敛电路上都能正常工作,但在该算法的SPICE实现中的一个缺陷会阻碍电源步进方法在其他电路上的收敛。

电源步进算法实际上是通过一种二进制下降的顺序把电路电源从满值向下形成阶梯状(满值,1/2满值,1/4满值,1/8满值,1/16满值)。在阶梯下降的阶段,SPICE一直在搜索一个收敛的偏置点来作为阶梯上升阶段的起始点。当偏置点被找到,SPICE用同样的二进制顺序来阶梯状提升电源。如果,在提升阶段,不收敛发生,算法会重复阶梯下降,阶梯上升的过程来工作通过不收敛。但是因为算法中存在的一个缺陷,在第二次阶梯下降阶段,电源会被保持在一个局部最小值,不会阶梯下降到这个点下面。无法使得电源进一步阶梯下降,SPICE很快会超过那一步被容许的迭代次数,进而中止直流偏置点计算。

对于大多数电路,提升ITL1和合理使用.NODESET声明会产生一个精确的偏置点解。当这些技术都失败了,电源步进算法才是一个好的备选的解技术。

\subsection{关闭有源元件}
SPICE包含的最后一个直流偏置点收敛辅助措施叫作关闭声明。关闭声明,像.NODESET声明一样,会强使仿真器进行两次直流偏置点计算。在第一次计算过程中,一个或者更多的有源器件会被关闭。在第二次计算过程中,有源器件会被开启,并且前一次计算的直流偏置点会被用作下一波Newton迭代的初值。

关闭声明可以加到电路中的一个或者多个半导体器件上。图\ref{图3.16}展示了SPICE中四种半导体类型的关闭声明的使用方法。

\begin{figure}[htbp]
\small
    \centering
    \includegraphics[width=0.5\textwidth]{figure/Chapter3/图3.16.png}
    \caption{SPICE半导体器件的关闭声明语法。}
    \label{图3.16}
\end{figure}

关于关闭声明的一句提醒:对于很多非线性电路来说,带有一个或者多个被关闭的有源器件的电路偏置点与这些器件在开启状态时的偏置点在本质上是不同的。出于这个原因,在应用了ITL1,.NODESET声明和ITL6技术后失败的电路上使用关闭声明很少能实现收敛。

\subsection{直流偏置点收敛性辅助总结}
直流偏置点不收敛问题可以通过遵循本节列出的步骤被明显解决掉。找到直流偏置点对仿真至关重要,因为直流偏置点计算是SPICE中其他任何分析的起始条件。出于这个原因,学会克服直流偏置点不收敛在产生精确的,高质量的仿真结果方面是最重要的。表\ref{表3.5}列出了直流偏置点收敛辅助措施。如果设置完广义Newton-Raphson收敛辅助措施后不收敛仍然持续,那么就按顺序遵循表\ref{表3.5}中的收敛辅助措施。

\begin{figure}[htbp]
\small
    \centering
    \includegraphics[width=0.7\textwidth]{figure/Chapter3/表3.5.png}
    \caption{直流偏置点收敛辅助措施}
    \label{表3.5}
\end{figure}

\section{直流扫描收敛辅助}
在许多方面,执行直流扫描分析就像在执行一系列的直流偏置点计算。分析开始前,SPICE会进行直流偏置点计算。一旦找到了偏置点,节点电压就会被保存,并且在分析中被用作下一个点的Newton迭代的初值。在分析中的每一个解点,SPICE会用前一次解作为下一波迭代的初值。这个过程在扫描分析中会为每一个点进行。当分析完成,SPICE打印输出结果。

但是与直流扫描分析相关的不收敛问题和直流偏置点的问题却令人惊奇地不同。直流扫描分析的两种不收敛机制是快速电压转换和模型不连续。无论它们中的哪个都会引起仿真器在找解的过程中失败。

\subsection{模型的不连续性}
SPICE用来仿真的半导体器件模型是模仿真实器件的物理行为。图\ref{图3.17}展示了MOS晶体管的曲线家族。经典物理把晶体管曲线切分成线性工作区和非线性饱和区。SPICE中用的器件方程遵循这些相同的工作区。但是,不同于真正的器件,SPICE对每个工作区用分离的方程。由于数学上难以写出描述整个曲线家族的方程,就写出了两套不同的方程,一个是线性区,另一个是饱和区,然后把它们两个合并在一起。不幸的是,因为方程被合并的方式,在器件的电导特性(I-V曲线的斜率)上存在不连续\cite{Vlach1983}。在图\ref{图3.18}中,在线性和饱和区的交界处存在不连续。

\begin{figure}[htbp]
\small
    \centering
    \includegraphics[width=0.5\textwidth]{figure/Chapter3/图3.17.png}
    \caption{MOS场效应晶体管的线性和饱和工作区}
    \label{图3.17}
\end{figure}

\begin{figure}[htbp]
\small
    \centering
    \includegraphics[width=0.5\textwidth]{figure/Chapter3/图3.18.png}
    \caption{Level 3 MOS场效应SPICE晶体管模型中的模型不连续。}
    \label{图3.18}
\end{figure}

电导上的不连续会给Newton-Raphson算法带来问题。图\ref{图3.19a}展示了在模型的一个不连续处电导vs.电压的特性。在第一个Newton迭代靠近不连续的一边,电导值导致在不连续的另一边的一个新的迭代电压值。在图\ref{图3.19b},下一次Newton迭代落到的电导值预测了一个解回到了不连续处的原始一边。在图\ref{图3.19c},第三次Newton迭代再一次在不连续距离较远的一边预测了一个解。当SPCIE一步一步靠近或者在模型不连续处的顶端时,Newton-Raphson迭代就会在不收敛周围开始震荡。这些震荡会用尽迭代次数但不会前进到解。

\begin{figure}[htbp]
\small
    \centering
    \includegraphics[width=0.5\textwidth]{figure/Chapter3/图3.19a.png}
    \caption{(a)在模型不连续处附近的迭代可能会从不连续的一边跳到另一边。}
    \label{图3.19a}
\end{figure}

\begin{figure}[htbp]
\small
    \centering
    \includegraphics[width=0.5\textwidth]{figure/Chapter3/图3.19b.png}
    \caption{(b)在模型不连续处附近的迭代可能会从不连续的一边跳到另一边。}
    \label{图3.19b}
\end{figure}

\begin{figure}[htbp]
\small
    \centering
    \includegraphics[width=0.5\textwidth]{figure/Chapter3/图3.19c.png}
    \caption{(c)在迭代过程中模型不连续可能会引发震荡。这种震荡会导致SPICE使用的迭代无法趋向于有价值的解。}
    \label{图3.19c}
\end{figure}

尽管当搜索直流偏置点解时模型不连续会引起问题,但在扫描分析中,模型不连续不收敛就不止一个问题了。进行直流偏置点计算时,只需要寻找一个解。在直流扫描分析中,需要很多解,而且在大多数分析中,非线性电路器件会被在一个或者多个工作区之间扫描。模型不连续只会在解步和不连续点对齐或者非常靠近时才会引起问题。扫描分析中使用的步数越多,改变步入或者靠近模型不连续的效果就越好。这就是为什么说相较于直流偏置点计算,在扫描分析中的模型不连续不收敛会给扫描分析带来不止一个问题的原因。

\subsection{快速电流转换}
在直流扫描分析中第二个不收敛机制是因为电压或者电流的快速转换。图\ref{图3.20a}展示了一个反相器的理论转换特性($V_out$ vs. $V_in$)。通过执行直流扫描分析,输入扫描从0到5 volts,可以得到转换特性并观察到输出电压。
\begin{figure}[htbp]
\small
    \centering
    \includegraphics[width=0.5\textwidth]{figure/Chapter3/图3.20a.png}
    \caption{(a)在直流扫描分析中,SPICE用前一次的电压解作为下一波Newton迭代的初值。}
    \label{图3.20a}
\end{figure}

为了启动分析,SPICE需要决定电路的直流偏置点。一旦找到偏置点,输入电压会被升至.DC声明中的第一个电压步,而且偏置点电压会被用来作为下一波Newton迭代的初值。在任何SPICE扫描型的分析中,前一个解电压都会被用作下一波Newton迭代的初值。使用偏置电压作为初值,SPICE会一直迭代下去直到找到第二个扫描点的解或者使用完ITL2次迭代。ITL2是在直流扫描分析每步中被容许的迭代次数。

在扫描分析中,这个过程会在每一个解点上重复。但是在大电压转换期间,扫描点之间的电压变化对Newton迭代来说太大了以至于找不到解,如图\ref{图3.20b}所示。在转换点不收敛是一种常见的直流扫描失效。
\begin{figure}[htbp]
\small
    \centering
    \includegraphics[width=0.5\textwidth]{figure/Chapter3/图3.20b.png}
    \caption{(b)在大电压转换期间,新解可能离之前的解很远。在这些条件下,SPICE经常无法收敛。}
    \label{图3.20b}
\end{figure}

当在电压转换点发生不收敛时,很多工程师会减小.DC声明中的步长(图\ref{图3.20c})。虽然减小直流声明中的步长会减小电压转换的间隔,并且通常可以得到成功的仿真,但是减小步长不是消除直流扫描电压转换不收敛失效的最优办法。

\begin{figure}[htbp]
\small
    \centering
    \includegraphics[width=0.7\textwidth]{figure/Chapter3/图3.20c.png}
    \caption{(c)减小直流扫描步长是一种解决电压转换不收敛问题的办法。但是,减小步长会增加仿真时间和增加遇到模型不连续的机会。提高直流扫描迭代限制(ITL2)是解决电压转换不收敛问题的最优方法。}
    \label{图3.20c}
\end{figure}

减小直流声明每步的增量会要求更多的解点。额外的解点需要额外的迭代,导致更长的仿真运行时间。如图\ref{图3.21ab}(a)和(b),减小步长会增加步入或者非常靠近模型不连续的地方以及因为震荡而导致的收敛失败的机会。

\begin{figure}[htbp]
\small
    \centering
    \includegraphics[width=0.7\textwidth]{figure/Chapter3/图3.21ab.png}
    \caption{(a)和(b)减小直流扫描步长会增加走向靠近模型不连续和收敛失败的机会。步长越小,遇到不连续的机会越大。}
    \label{图3.21ab}
\end{figure}

在分析中提高每步被容许的迭代次数ITL2是最好的克服快速电压转换不连续的办法。在直流扫描分析中,ITL2限制了每个解点上的迭代次数。默认地,ITL2被设置为50。对很多电路来说,穿过大的快速电压转换区需要超过50次迭代。在快速电压转换中,也许需要100,200或者更多次的迭代才能定位解。在直流扫描分析中,对于包含一次或者多次陡坡转换的电路,需要在.OPTIONS声明中把ITL2提升到200(.OPTIONS ITL2 = 200)。

如果,在直流扫描分析中,提升ITL2不能纠正不收敛问题,那么不收敛问题就可能是因为一个或者多个模型的不连续。在这种情况下,用户可以选择要不获取一个带有最小化不连续的模型参数值的新模型,要不增加或偏置分析步长。通常,通过增加或者偏置分析步长会克服模型不连续,因为这样会使得步子足够远离不连续的地方以避免震荡。图\ref{图3.22}展示了用户如何增加或者偏置直流扫描分析步长。

\begin{figure}[htbp]
\small
    \centering
    \includegraphics[width=0.7\textwidth]{figure/Chapter3/图3.22.png}
    \caption{改变分析步长以克服不收敛。}
    \label{图3.22}
\end{figure}

\subsection{直流扫描收敛总结}
直流扫描分析不收敛失效经常由两种机制引起,快速电压转换或者器件模型不连续。克服这些问题的办法总结在了表\ref{表3.6}。
\begin{figure}[htbp]
\small
    \centering
    \includegraphics[width=0.7\textwidth]{figure/Chapter3/表3.6.png}
    \caption{直流扫描收敛辅助措施。}
    \label{表3.6}
\end{figure}

\section{交流扫频收敛辅助}
相比于SPICE中的任何其他分析类型,交流扫频分析不太容易遇到不收敛问题。原因是(如第\ref{chap:2}章所述)交流扫频是一种小信号分析,当找到偏置点后,分析中就不会再包含任何非线性行为。SPICE首先确定电路的直流偏置点;在计算偏置点的过程中,SPICE有可能收敛失败。在这些例子中,直流偏置点收敛辅助会被用来完成收敛。偏置点被找到后,非线性器件模型会被它们的线性小信号模型等效电路全部替换。电路中没有非线性器件,SPICE会转换到比较简单的LU分解(高斯消去)算法。有了LU分解求解器,求解电路就不需要Newton迭代了,而且正因为此,不收敛也会消失。一旦交流扫频分析决定了电路的直流偏置点,扫频分析通常都会收敛。唯一与交流扫频相关的不收敛问题来自于寻找电路的直流偏置点,而这些问题在名为“直流偏置点收敛辅助”一节里已经讨论过。

\section{瞬态收敛辅助}
对很多SPICE用户来说,瞬态分析是使用最频繁的分析类型。在很多方面,瞬态不收敛问题和直流扫描分析不收敛问题相似,但是有些方面它们两个是不同的。

开始瞬态分析之前,SPICE会开路电路电容,短路电路电感,然后计算电路的直流偏置点。一旦偏置点被找到,电容和电感会被立刻恢复到电路中,并且假设电压是由直流偏置点计算建立来的(不要求充电时间)。当电荷存储元件被放置到电路中后,第一个时间点会被计算,在分析的第一个时间点上计算解电压和电流的一系列Newton迭代将开始。在每一个新解(时间点),一波新的Newton迭代会被用来计算电路的节点电压和支路电流。随着解算法迭代,数值积分算法把电容电流和电感电压确定为时间的函数。所有这些都发生在电路直流偏置被找到之后的每一个时间点上。

\subsection{动态时间步长控制}
像直流扫描分析一样,不收敛问题主要来自于快速电压转换和器件模型不连续。但是,与直流扫描分析不同,SPICE用一种动态时间步长控制算法来计算解,而不是通过用户专门指定。这对很多用户来说是一个惊喜。分析声明

.TRAN 1NS 100NS

是一个指导SPICE从时间T=0到T=100nS执行瞬态分析的指令。但是声明的第一个参数,1nS,叫作打印间隔,是分析每1nS打印一次输出结果的指令。SPICE并不是在分析的每1nS就会求解一次电路。内部的时间步长控制算法负责为分析选择真正的解时间点。

在瞬态分析过程中,SPICE把所有的解都存在内存中。分析完成后,SPICE用一个线性插值函数来均匀地计算输出点。图\ref{图3.23}展示了内部(解)时间点和打印结果之间的差别。(RSPICE的NOINTR选项强制RSPICE打印这些内部时间点,而不是插值时间点。)
\begin{figure}[htbp]
\small
    \centering
    \includegraphics[width=0.7\textwidth]{figure/Chapter3/图3.23.png}
    \caption{SPICE在瞬态仿真过程中使用的时间点与输出文件中打印出的点不一样。很多仿真器提供选项以便用户可以看到SPICE用来求解电路的真正时间点。}
    \label{图3.23}
\end{figure}

动态时间步长控制算法根据电路活动改变时间步长的尺寸。在高电路活动时期,大的电压和电流变化,为了帮助增加精度和减少因为快速电压转换带来的不收敛,时间步长控制算法会保持一个小的时间步长。在低电路活动时期,或者很少或者无电压和电流变动,为了加速仿真得到结果,时间步长控制算法会增加时间步长。

动态时间步长控制算法也使得瞬态分析可以做一些其他分析做不了的事情。如果SPICE在一个时间点上收敛失效,那么时间步长控制算法会自动地丢弃该不收敛的时间点,把时间步长降低为原来尺寸的$\frac{1}{8}$,并且用该较小的时间步长来重新尝试求解。如果SPICE再一次失败,那么时间步长就会被再一次消减一个8倍的因子,这样一遍一遍地直到电路收敛了或者时间步长被减到了电路容许的最小值。(最小内部时间步长是内部定死的,无法被用户调整。)如果内部时间步长被减到了容许的最小值之下,SPICE将打印“内部时间步长太小”的警告信息(Pspice用户:Pspice用打印“在瞬态分析中不收敛”来替换这条信息。)和中止仿真。SPICE中的每一个其他分析类型在第一个不收敛解上中止仿真。

SPICE的作者们承认在瞬态分析过程中,快速电压转换不收敛会成为一个问题。正是出于这个原因,在SPICE中实施了动态时间步长控制算法。

\subsection{初始条件}
瞬态分析因为新的初始化技术,.IC声明,也与其他任何分析类型不同。在瞬态分析中,.IC声明被用来设置电路的初始条件。.IC声明的语法如下所示。

.IC V(3)=2.4 V(5)=7.6 V(13)=2.8

.IC声明与本章早些时候讨论的.NODESET声明相似。.IC和.NODESET都是用一个Norton等效电压源建模的。图\ref{图3.24}展示了.IC声明的Norton等效电路。但是这也是.NODESET声明和.IC声明之间相似性的终点。
\begin{figure}[htbp]
\small
    \centering
    \includegraphics[width=0.7\textwidth]{figure/Chapter3/图3.24.png}
    \caption{SPICE中用来建模.IC声明电压的Norton等效电压源。}
    \label{图3.24}
\end{figure}

.IC声明只在瞬态分析中有效,而.NODESET声明在所有分析类型中都是有效的。在直流偏置点,直流扫描,交流扫频分析中,SPICE在输入文件中忽视任何.IC声明。如果,然而,在瞬态分析中,.NODESET声明和.IC声明都被包含进了输入文件,那么只有.IC声明会被用到;.NODESET声明会被丢弃。

另一个.NODESET和.IC声明之间的差别包括SPICE如何决定偏置点。在直流分析中,.NODESET Norton等效电源会被一直保持在电路中直到找到一个偏置点。随后SPICE会移除电源并执行第二次直流偏置点计算,这样的话Norton等效电源就不会参与到真正的电路中。与.NODESET声明不同,.IC Norton等效电源只在一次系列迭代中被保持。当找到一个稳定的偏置点时,SPICE用该偏置点作为电路的初始瞬态条件。有时候,对于一些电路这种做法会引起问题,因为偏置点建立时Norton等效源仍然在电路中。

为了电荷电路节点或者为了启动不稳定的电路,.IC声明经常被用来设置初始条件。图\ref{图3.25}是一个简单的RC电路,其中电容的初始电压是1V。为了仿真输出电压中的RC衰减,瞬态分析或许可能会被执行。用命令仿真硬盘文件ch3-26a.cir:

SIM CH3-26A.CIR

\begin{figure}[htbp]
\small
    \centering
    \includegraphics[width=0.7\textwidth]{figure/Chapter3/图3.25.png}
    \caption{用一个简单的电路来说明.NODESET和.IC初始条件声明之间的不同。}
    \label{图3.25}
\end{figure}

你的结果应该和图\ref{图3.26a}中的匹配。在ch3-26a.cir电路文件中,.NODESET声明被用来设置电容的初始1V电压。使用.NODESET声明的问题是当找到一个稳定的偏置点后,Norton等效电压源被移除了。在ch3-26a.cir电路中,SPICE很快建立起1V作为初始直流的条件。但是当找到一个初始偏置点后,SPICE移除了.NODESET电压源,并且执行了第二波迭代。在ch3-26a.cir电路文件中,没有了Norton等效电压源,电阻两端的电压立刻将为0.(记住,直到偏置点被确定,电容是不会被重新加入电路的。)正因为初始化电荷电路节点的需求,SPICE的作者们才加入了.IC声明。用命令仿真ch3-26b.cir电路文件:

SIM CH3-26B.CIR

你的结果应该和图\ref{图3.26b}匹配。

电路文件ch3-26b.cir和ch3-26a.cir除了用.IC声明取代了原文件中的.NODESET声明外其他部分完全一样。

\begin{figure}[htbp]
\small
    \centering
    \includegraphics[width=0.7\textwidth]{figure/Chapter3/图3.26a.png}
    \caption{图\ref{图3.25}中电路的输出响应。.NODESET声明被用来设置电容的初始电压。(重印自Successfully Simulating Circuits with SPICE。授权使用。)}
    \label{图3.26a}
\end{figure}

\begin{figure}[htbp]
\small
    \centering
    \includegraphics[width=0.7\textwidth]{figure/Chapter3/图3.26b.png}
    \caption{图\ref{图3.25}中电路的输出响应。.IC声明被用来设置电容的初始电压。(重印自Successfully Simulating Circuits with SPICE。授权使用。)}
    \label{图3.26b}
\end{figure}

\subsection{瞬态分析中的不收敛}
有了直流扫描分析和瞬态分析之间的区别定义,瞬态分析的不收敛问题就可以被检查了。与瞬态分析相关的两种决定性不收敛问题是快速电压转换和器件模型不连续。这些都是直流扫描分析中的相同的两种失效机制。

在直流扫描分析中,快速电压转换不收敛可以通过增加分析中每步被容许的迭代次数或者减小分析的步长来纠正。在瞬态分析中,动态时间步长控制算法会在大的电路转化时期自动地减小步长。在直流扫描分析中,器件模型不连续不收敛可以通过增加步长来克服,但是在瞬态分析中,动态时间步长控制算法会决定步长。

在瞬态分析中,当电路趋向于电压转换点,两种潜在冲突的事件会发生。第一,时间步长控制算法会减小时间步长的尺寸,第二,在电压转换时期,电路的半导体器件将改变状态,从一个工作区变换到另一个。因为半导体器件在工作区之间变化,一个或者多个模型不连续就会暴露。瞬态分析中电压转换会导致时间步长被减小,这就会使得在这段时间器件模型不连续有很大的可能无法被覆盖。更糟糕的是,如果不连续引发了不收敛,时间步长控制算法会自动地减小步长的尺寸,而且被减小的步长几乎肯定会再次在同一个不收敛处出错,进而又发生收敛失败。(在直流扫描不收敛那节,通过使用增加步长尺寸来帮助跨过不连续区的办法,模型不连续可以很好地被克服。)

\subsection{器件模型电容}
瞬态不收敛主要是由模型不连续和电路中电压转换带来的明显减小的步长的组合引起的。这种失效机制可以通过首先认识到模型不连续是在器件模型的直流I-V特性中得到纠正。所有器件模型都有一个内建电容。内建电容要不是与器件PN结相关的结电容,要不就是与MOSFET的绝缘栅相关的覆盖电容。通常这些电容性器件电流会帮助连接直流不连续。不幸的是,内建电容都有一个默认的0值,如果存在直流不连续,就会引起收敛问题。所有真实的半导体器件都有一些电容性元件;因此,所有仿真模型本应当给与它们相关的电容项设置一个非零值。给模型加电容项可以同时帮助提升瞬态分析的精度和仿真器的收敛性。表\ref{表3.7}展示了每一种半导体器件的电容模型参数。

\begin{figure}[htbp]
\small
    \centering
    \includegraphics[width=0.7\textwidth]{figure/Chapter3/表3.7.png}
    \caption{电容模型参数}
    \label{表3.7}
\end{figure}

如果器件参数中不包含电容项,要不为电容找一个精确的值并把该值加入到模型中,要不给电容参数设置一个足够小的值以防止对电路的正常操作产生影响。即使一个小的电容性元件经常也是足以克服不收敛问题的。表\ref{表3.8}同时展示了分立和集成电路器件的每个电容项的通用最小值。对一些电路来说这些值太大了。用表\ref{表3.7}时小心。

\begin{figure}[htbp]
\small
    \centering
    \includegraphics[width=0.7\textwidth]{figure/Chapter3/表3.8.png}
    \caption{最小电容参数设置}
    \label{表3.8}
\end{figure}

图\ref{图3.27}展示了一个简单的SR触发器。触发器从CMOS等效门构建而来。电路文件也包含着两个脉冲电压源,一个在设置引脚,一个在复位引脚。仿真文件对触发器执行瞬态仿真,通过几个状态变化驱动闩锁。用命令仿真ch3-27.cir文件:

SIM CH3-27.CIR

电路会在第一个电压转换点收敛失败。


\begin{figure}[htbp]
\small
    \centering
    \includegraphics[width=0.7\textwidth]{figure/Chapter3/图3.27.png}
    \caption{一个与非门触发器}
    \label{图3.27}
\end{figure}

现在仿真ch3-28a.cir电路文件。除了加入了MOSFET电容项CGBO, CGSO, CGDO, CBD,和CBS外,该文件与ch3-27.cir文件完全一致。电容项由表\ref{表3.7}决定。

SIM CH3-28A.CIR

你的结果应该和图\ref{图3.28a}匹配。在这个电路中,加了一个小的电容项就消除了不收敛失效。

\begin{figure}[htbp]
\small
    \centering
    \includegraphics[width=0.7\textwidth]{figure/Chapter3/图3.28a.png}
    \caption{(a)电路文件中包含电容模型参数的SR触发器的瞬态响应。(重印自Successfully Simulating Circuits with SPICE。授权使用。)}
    \label{图3.28a}
\end{figure}

\subsection{提高迭代限制}
在直流扫描分析一节中,提高迭代限制容许仿真器在中止仿真前进行更多次的迭代。对瞬态分析,迭代限制是ITL4。如果一个时间点在迭代ITL4次后仍然收敛失败,那么SPICE就会舍弃该点,削减步长8倍,然后用一个更小的时间步长来重新求解。通过提高ITL4,SPICE会有更多的迭代机会来收敛到解。这样可以减少SPICE被强制缩小时间步长的次数,而且也会降低SPICE运行进入或者靠近模型不连续进而收敛失效的机会。

默认地,ITL4只被设置了10次迭代。对于所有的瞬态电路,在.OPTIONS声明(.OPTIONS ITL4=40)中提升ITL4到40次迭代。

用命令仿真ch3-28b.cir电路文件:

SIM CH3-28B.CIR

你的结果应该和图\ref{图3.28b}匹配。ch3-28b.cir电路文件包含原始的SR触发器,不带电容项但是把ITL4参数提高到了40。注意从这个电路中取得的结果与之前的结果完全一致,但是ch3-28b.cir的运行时间比ch3-28a.cir的一半还要少。

\begin{figure}[htbp]
\small
    \centering
    \includegraphics[width=0.7\textwidth]{figure/Chapter3/图3.28b.png}
    \caption{(b)瞬态迭代限制(ITL4)设置为40的SR触发器的瞬态响应。(重印自Successfully Simulating Circuits with SPICE。授权使用。)}
    \label{图3.28b}
\end{figure}

提高ITL4的一个附加的好处是对仿真速度的提升。第\ref{chap:5}章会展示SPICE中的时间步长控制算法经常用8倍的因子来削减时间步长,但是时间步长控制算法增加时间步长的因子却不超过2。如果SPICE收敛失败并且削减了时间步长,那么在回到时间步长被削减的原来的仿真时间点上必须得求解3个以上的时间点。对很多电路来说,提高ITL4会提高仿真器的速度并且减少不收敛的发生。

\subsection{瞬态收敛辅助总结}
克服瞬态不收敛失效要求理解直流偏置点和直流扫描分析的不收敛失效。动态时间步长控制算法帮助减少因为快速电压转换引起的不收敛失效,但是也许同时会因为模型不连续而加剧失效。把器件模型中的电容项从它们默认的0值恢复到一个更实际的值有助于减少模型不连续。提高ITL4可以给时间步长算法在时间步长被减少前更多的时间来收敛到解。保持一个较大的时间步长可以降低步入或者靠近模型不连续的机会。一起使用的话,这两种技术会消除大部分的瞬态不收敛失效。表\ref{表3.9}概括了瞬态收敛辅助以及它们应该被使用的先后顺序。

\begin{figure}[htbp]
\small
    \centering
    \includegraphics[width=0.7\textwidth]{figure/Chapter3/表3.9.png}
    \caption{瞬态收敛辅助。}
    \label{表3.9}
\end{figure}

\section{不收敛总结}
用户面对的大多数常见的不收敛问题可以通过SPICE中的可行的控制和选项来克服。许多问题只是简单地发生因为误差容限和程序的默认值对给定的电路不合适(表\ref{表3.2})。一旦这些设定了,特定的分析类型可能会引起问题。表\ref{表3.5}展示了计算电路直流偏置点的收敛辅助。表\ref{表3.6}展示了直流扫描分析的收敛辅助。最后,表\ref{表3.9}展示了瞬态收敛辅助。

第一,设置广义的收敛辅助措施。当不收敛发生时,确定是哪一种分析引起了不收敛,然后根据这种类型查找特定的表。不要迷惑收敛辅助。设置直流偏置点辅助不会帮助解决瞬态不收敛问题。在直流扫描中,设置瞬态辅助也不会帮忙。在交流分析中,设置直流扫描辅助也不会得到帮助。识别出哪种分析引起了不收敛,并且学会哪种收敛辅助措施可以应用到每一种分析类型。

\section{在类SPICE仿真器中的收敛性辅助}
\subsection{Hspice}
Hspice和标准的SPICE有许多相同的收敛辅助,也包括几种加强项。收敛辅助选项包括GMIN, RELTOL,ABSTOL,,VNTOL,ITL1,ITL2,和ITL4。加强项包括电源步进,GMIN-斜坡法,和伪-瞬态偏置决定。

当选项CONVERGE=3时,Hspice用一种修正的电源步进算法。Hspice用一种可变的步长算法取代SPICE中的二进制步进算法。

当Hspice选项GRAMP=X被指定时,Hspice调用GMIN-斜坡算法。GMIN-斜坡法包括给GMIN电导(电阻)设置一个大的(小的)值。电路中给每个非线性器件接一个大电导(小电阻)的话,很多电路的非线性行为会被抑制。在这些条件下,HSPICE收敛非常快。收敛之后,Hspice用10倍的因子降低GMIN的值,然后重新尝试求解。这个过程一致持续到电路收敛失败或者GMIN被减到默认设置的值。如果Hspice收敛失败,GMIN会被增加10倍,并在接下来的仿真中得到保持。

为了启动斜坡法,GRAMP=X参数会被用来设定GMIN电导的指数值。例如,如果.OPTION GRAMP=6被使用,Hspice会从1e-6到1e-12 mhos (1e6 ohms 上升到1e12 ohms)斜坡状调整GMIN电导(电阻)。

为了找到电路的直流偏置点,Hspice加入一种叫作伪-瞬态分析的技术。有了伪-瞬态分析,电路中每个节点会被附加上一个小电容。通过从0升至满值地斜坡状调整直流电源,分析会一致进行下去。一旦电源处于满值状态,电容会被从电路中移除,而结果就是合适的直流偏置点。Hspice的选项CONVERGE=1唤醒伪-瞬态分析。

\subsection{IS\_{Spice}}
IS\_{Spice}和标准的SPICE有着同样的收敛辅助。收敛辅助选项包括GMIN,RELTOL,ABSTOL,VNTOL,ITL1,ITL2,ITL3,ITL4,和ITL6。

\subsection{Micro-Cap IV}
Micro-Cap IV和标准的SPICE在很多方面有着相同的收敛辅助,另外加上SAVE BIAS命令。收敛辅助选项包括GMIN,RELTOL,ABSTOL,VNTOL,ITL1,ITL2,ITL3,和ITL4。

虽然不是直接的收敛辅助措施,但是SAVE BIAS命令容许用户从之前的仿真中复用偏置点电压。对于大电路,通过消除消耗在偏置点计算中的时间和迭代,复用偏置点会加速随后的运行速度。

\subsection{PSpice}
Pspice和标准的SPICE在很多方面有着相同的收敛辅助,另外加上三个额外的加强项。收敛辅助选项包括GMIN,RELTOL,ABSTOL,VNTOL,ITL1,ITL2,和ITL4。收敛加强项包括给独立电流源,受控电压和电流源加额外的GMIN电阻,对直流偏置点和直流扫描计算进行自动电源步进,对.NODESET和.IC声明加上一个减掉的电阻,以及SAVEBIAS命令。

GMIN电阻会被接在电路中的每个独立电流源和每个受控源上。在.OPTIONS声明中改变gmin的值可以调整这个值。

在直流偏置点计算中如果Pspice在ITL1次迭代后收敛失败,或者在直流扫描分析中如果Pspice在ITL2次迭代后收敛失败,那么程序会自动切换到电源步进算法。Pspice中的电源步进算法用了可变的步长,而不是标准SPICE中的二进制步进方法。

在SPICE中,.NODESET和.IC声明用Norton等效源来建模。Norton等效源包含一个电流源和一个1-ohm的电阻。对很多低阻抗电路,1-ohm电阻会给电路加上明显的负载。在Pspice中,1-ohm电阻被一个.002-ohm的电阻取代,同时电流源的值会增加以匹配减小的电阻。

虽然不是直接的收敛辅助措施,SAVEBIAS命令容许用户从之前的放置中复用偏置点电压。对于大电路,通过消除消耗在偏置点计算中的时间和迭代,复用偏置点会加速随后的运行速度。

\section{总结}
不收敛是仿真用户面对的一种最常见和令人沮丧的问题。解决不收敛问题包括识别和消除不收敛的起因。在你仿真的每一个电路中设置广义收敛辅助。如果不收敛发生,识别出是哪种分析类型引起了不收敛,并设置特定分析的收敛辅助。遵循这里概括出的指导原则可以解决所有不收敛问题中的80\%到90\%。

一些不收敛问题也许会比其他问题更顽固。对于困难的问题,准确遵循给出的指导原则。有时候,你尝试的第一个收敛辅助就会解决问题。其他问题会要求所有的收敛辅助。不要担心加了太多的收敛辅助。在你的仿真中如果遵循了本章提供的指导原则,你不会引入任何可观的误差。这里给出的指导原则代表了一种经过证明的系统的解决不收敛问题的技术。

克服不收敛问题是可能的和可行的。在仿真精度可以确定之前,在仿真速度可以提升之前,在分析结果可以检验之前,电路必须要先能收敛。

\bibliographystyle{plain}
\bibliography{Reference/chapter3-reference}