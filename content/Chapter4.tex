\chapter{数值积分}
\label{chap:4}
什么是数值积分?SPICE中为什么要使用数值积分?这两个重要的问题对任何有知识的仿真用户是应该能回答的。如果你不知道这些问题的答案,往下读。太少的用户意识到数值积分在产生精确的瞬态分析时的重大作用。

作为时间的函数,数值积分被用在SPICE中来计算流过电路电容的电流;作为对偶作用,也是时间的函数,数值积分被用来计算电路电导的压降。

至于什么是数值积分,数值积分是以一种分段线性的方式施加的解析积分。当解析积分产生代表解的连续函数,数值积分产生离散的解。如果把所有的解画在图上,结果曲线会与解析积分产生的解匹配。

大多数时候,数值积分被用来求解有一个未知数或者难以计算解析解的微分方程。在电路仿真中,感兴趣的微分方程是
\begin{equation}
    I=\frac{C*dV}{dt}
    \label{eq:4.1}
\end{equation}
和
\begin{equation}
    V=\frac{L*dI}{dt}。
    \label{eq:4.2}
\end{equation}
通过求解与电压和电流相关的微分方程,数值积分确定电容电流和电感电压。

在瞬态分析中,SPICE从三种用户可以选择的积分方法-trapezoidal,backward-Euler,或者Gear法-中使用一种来确定电容电流和电感电压。除了标准的Gear积分公式,Gear方程的四种额外变体也可以被选择。这给了SPICE用户一个很大的积分方法选择范围。

\section{瞬态仿真警告}
SPICE中之所以提供了一系列的积分方法简单地说是因为没有一种积分方法对于所有的瞬态仿真总是最好的。当数值积分公式被应用到微分方程,积分产生的分段线性解只是准确方程的一个近似。在每一个解上,积分算法可能会引入一小部分的误差。数值积分方法的精度依赖仿真时间步长尺寸,电压的形状和电流的波形,以及使用的积分方法。因为瞬态分析会产生很多不同的波形,为了在不同电路响应下保持精度,数值积分算法必须极其强健。一种给定的积分方法也许在一个函数或者波形上工作地很好,但在另一个电路上会产生完全不准确的结果。因为没有一种积分方法对所有的瞬态仿真都适应的最好,所以SPICE用户必须能够决定哪种积分方法对他们的电路可以产生最精确的结果。

当使用不当,数值积分不精确对仿真用户来说就是Achilles的脚踝。与不收敛失效不同,当数值积分方法产生不精确结果时SPICE不会产生警告信息。SPICE用户必须学会在仿真中寻找一些事情已经出错的指示。经常,当积分方法失效时,仿真结果会表现出定义好的,特征好的异常行为。积分失效可以表现为灾难性的失效,也可以轻微地如同电压或者电流的涟漪。但是对训练过的眼睛,这些相同的提示就是需要切换一种不同的积分方法的信号。学会数值积分失效的特征提示和纠正失效需要做的事情,是仿真手艺里两门重要的课。

\subsection{警告总结}
仿真用户应该不要假定从仿真得来的结果总是正确的。关于这方面有两个原因:第一,模型不精确会使得仿真器产生不精确的经过;第二,即使模型完全准确,仿真器中的几种失效机制也可能不警告用户引起错误的解(数值积分失效只是失效机制中的一种)。从来不要简单地认为仿真结果是正确的。如果出现有问题的结果,用好的工程判断来决定异常是仿真器相关的还是设计相关的。从来不要在对仿真输出应该是怎么样没有一个理性的判断前仿真电路。好的设计人员从不会在没有对电路行为有一些合理的预期之前来用面包板搭建一套电阻,电容,电感和晶体管。仿真用户对计算机仿真应该持有同样的态度。

\section{数值积分}
在前进之前,这里的一个数值积分的简单例子或许有用。一个时间函数如方程(\ref{eq:4.3})所示。
\begin{equation}
    F(t)=2t
    \label{eq:4.3}
\end{equation}

方程\ref{eq:4.3}的导数可以轻易得到,如方程(\ref{eq:4.4})所示。
\begin{equation}
    F^{'}(t)=2
    \label{eq:4.4}
\end{equation}
导数$F^{'}(t)$是一个常数,并且是通过定义微分方程得到的。方程(\ref{eq:4.4})作为我们例子的起始点。

方程(\ref{eq:4.4})是一个微分方程。微分方程就是一个函数的导数。通常原始函数是未知的,但是在这个例子中,导数和原始函数都已知。求解一个微分方程包括把微分方程转换回原始方程。

方程(\ref{eq:4.4})通过分离变量法用解析技术也可以求解。通过用微分方程的形式重写方程(\ref{eq:4.4})。
\begin{equation}
    \frac{dY(t)}{dt}=2
    \label{eq:4.5}
\end{equation}

给方程(\ref{eq:4.5})两边同时乘以$dt$可以得到方程(\ref{eq:4.6})。
\begin{equation}
    dY(t)=2*dt
    \label{eq:4.6}
\end{equation}

方程(\ref{eq:4.6})在时间上积分产生方程(\ref{eq:4.7})。
\begin{equation}
    Y(t)=2t+C
    \label{eq:4.7}
\end{equation}

不用惊奇,方程(\ref{eq:4.7})和(\ref{eq:4.3})除了常数$C$以外都完全一致。

分离变量法是一种解析的方法来求解微分方程。方程(\ref{eq:4.5})也可以用数值积分来解。

数值积分产生的结果与解析解相似但是以一种分段线性的方式。注意使用的词是“相似”,而不是“一致”。数值积分是解的一种近似。在每一个解处,一部分有限误差会被引入。经常,误差是不明显的,数值积分解非常接近解析解,但是有时候积分误差也可能明显。

在数值积分过程中,原始函数会被在积分的每一步上重新构建成一系列离散的数值点。为了给方程(\ref{eq:4.5})施加数值积分,如图\ref{图4.1}构建X-Y坐标轴网格。因为方程(\ref{eq:4.4})是时间的函数,时间的离散单位将是积分的间隔。
\begin{figure}[htbp]
\small
    \centering
    \includegraphics[width=0.4\textwidth]{figure/Chapter4/图4.1.png}
    \caption{X-Y坐标轴网格。}
    \label{图4.1}
\end{figure}

为了给微分方程施加数值积分,必须选定一种积分公式,另外一个起始点或者解的初值必须知道。对于这个例子,使用backward-Euler数值积分公式,Y(t=0)会被设置为0。backward-Euler公式如方程(\ref{eq:4.8})所示。
\begin{equation}
    Y(x+1)=Y(x)+step size * \frac{dY(x+1)}{dx}
    \label{eq:4.8}
\end{equation}

当给方程(\ref{eq:4.5})施加backward-Euler公式后,可以得到方程(\ref{eq:4.9})。
\begin{equation}
    Y(t+1)=Y(t)+timestep * \frac{dY(t+1)}{dt}
    \label{eq:4.9}
\end{equation}

在方程(\ref{eq:4.9})中,$Y(t+1)$表示在时间点$T+1$微分方程的解。$Y(t)$是过去时间点$T$处的解。对于我们的例子,时间步长会被设置为一个恒定的.5秒。积分从设置$Y(t=0)$为0和在离散的.5秒时间点处评估方程(\ref{eq:4.9})开始。表\ref{表4.1}展示了在每一个时间点上的计算过程和得到的值。图\ref{图4.2}重构了$Y(t)$图像。
\begin{figure}[htbp]
\small
    \centering
    \includegraphics[width=0.4\textwidth]{figure/Chapter4/表4.1.png}
    \caption{计算好的时间点上的解}
    \label{表4.1}
\end{figure}

\begin{figure}[htbp]
\small
    \centering
    \includegraphics[width=0.4\textwidth]{figure/Chapter4/图4.2.png}
    \caption{数值积分结果。}
    \label{图4.2}
\end{figure}

很多实际的工程问题必须得用微分方程来解。数值积分的目标是从已知的微分方程重构原始函数。在积分的每一步,积分公式产生一个离散的解。原始函数用一系列解来表示。

在这个例子中,数值积分解与解析解完全相等,但是这只是因为这个简单函数的微分方程是常数。在更多复杂的函数中,比如一阶导不是常数或者二阶导非0,很多数值积分方法在积分的每一步会产生一定量的小误差。

\section{SPICE和数值积分}
在SPICE中,数值积分函数求解描述电容和电感的电流电压关系的微分方程。方程(\ref{eq:4.10a})是电容I-V特性的backward-Euler表达式。关于电感的对偶表达式也可以写出。
\begin{equation}
    V_c(t+1)=V_c(t)+\frac{timestep*I_c(t+1)}{C}
    \label{eq:4.10a}
\end{equation}

为了从方程(\ref{eq:4.10a})确定电容电流,电压项和电流项必须出现在等式的不同端。使用简单的代数运算重新排布方程(\ref{eq:4.10a})中的项得到一个关于电容电流的更熟悉的表达式。
\begin{equation}
    \frac{[V_c(t+1)]-V_c(t)}{timestep}*C = I_c(t+1)
    \label{eq:4.10b}
\end{equation}

在瞬态仿真的每一个时间点,目前解$V_c(t+1)$和前一时刻$V_c(t)$的节点电压被用来确定电容电压的变化。
\begin{equation}
    \frac{[V_c(t)-V_c(t-1)]}{timestep}=\frac{dV_c}{dt}
    \label{eq:4.11}
\end{equation}

一旦电压改变量知道,电容电流可以通过给方程(\ref{eq:4.11})乘以电容值来确定。
\begin{equation}
    \frac{C*[V_c(t+1)-V_c(t)]}{timestep}=\frac{C*dV_c}{dt}=I_c(t+1)
    \label{eq:4.12}
\end{equation}

当SPICE开始瞬态仿真,对电路中的每个电容都会产生一个类似方程(\ref{eq:4.12})的方程,同时对电路中的每个电感都会产生一个对偶方程。

在寻找解的过程中,SPICE在节点电压值上一遍一遍地迭代直到找到一组电压满足Kirchhoff电压和电流定律,并且使电路中的每个电容/电感都满足方程(\ref{eq:4.12})。因为$V_c(t)$是前一时刻的节点电压(不是来自前一次迭代)并且保持固定,所以只有$V_c(t+1)$会随着每一个新的迭代电压值改变。$V_c(t+1)$变化的时候,$I_c(t+1)$也会变化着来匹配新的电容电压。当准确解找到时,方程(\ref{eq:4.12})就成为瞬态时间点处电容电压和电流之间的关系。

\section{数值积分的类型}
在瞬态仿真中,SPICE提供给用户三种不同的数值积分方法使用。这三种方法是backward-Euler,trapezoidal和Gear积分。在这些方法中,backward-Euler和trapezoidal方法都是17世纪中期\cite{Euler}就发展起来的。只有Gear方法是本世纪\cite{Gear}开发的。这些不同的积分方法都被写进了SPICE中,因为Nagel\cite{Nagel}发现没有一种方法对于所有的瞬态仿真都是可靠的。对一些电路,trapezoidal方法产生最精确的结果;对另外一些电路,Gear方法工作地最好;还有一些电路只有backward-Euler方法能产生最好的结果。

尽管每种积分方法必须执行同样的任务,但是每种方法产生解的过程却不同。对于给定的电路类型,理解每种方法如何完成求解电容电流和电感电压的任务,知道方法之间的区别,可以帮助选择一种合适的方法。

虽然SPICE只有三种不同的积分类型,但是为了说明方法间的不同和相似性,这里会提到四种。这四种感兴趣的类型是forward-Euler,backward-Euler,trapezoidal,和Gear积分。为了阐明方法间的不同,有两套X-Y坐标轴格点的图形技术会被使用。第一个图形的Y轴表示微分方程$dY(t)/dt$。第二个图形同时表示精确解和数值积分解。第二个图中的解析解用实线表示。在图中数值积分解用一系列的数据点表示。

图\ref{图4.3}表示了forward-Euler积分方法。通过把前一个时刻的解$Y(t)$加到时间步长与时间点$T$处的导数$dY(t)/dt$的积上,forward-Euler估计函数$Y(t+1)$的值。forward-Euler的名字来自于在建立$T+1$时刻解的导数时采用了前一个时间点$T$处的导数。感觉上,forward-Euler方法是基于前一个解的导数来评估当前解。forward-Euler积分公式如方程\ref{eq:4.13}所示。
\begin{equation}
    Y(t+1)=Y(t)+timestep*dY(t)
    \label{eq:4.13}
\end{equation}

\begin{figure}[htbp]
\small
    \centering
    \includegraphics[width=0.5\textwidth]{figure/Chapter4/图4.3.png}
    \caption{Forward-Euler数值积分公式和该方法的图形表示。}
    \label{图4.3}
\end{figure}

图\ref{图4.4}展示了backward-Euler积分方法。forward-和backward-Euler方法非常相似。两种方法的差别就在于选择从哪个时间点评估函数的导数。通过把前一时刻的解$Y(t)$加到时间步长与时间点$T+1$处的导数$dY(t+1)/dt$d 积上,backward-Euler方法估计函数$Y(t+1)$的值。当forward-Euler方法用前一时刻解的时间点$T$处的导数时,backward-Euler方法用当前时刻解的时间点$T+1$处的导数。backward-Euler积分公式如式\ref{eq:4.14}所示。
\begin{equation}
    Y(t+1)=Y(t)+timestep*dY(t+1)
\end{equation}

\begin{figure}[htbp]
\small
    \centering
    \includegraphics[width=0.5\textwidth]{figure/Chapter4/图4.4.png}
    \caption{Backward-Euler数值积分公式和该方法的图形表示。}
    \label{图4.4}
\end{figure}

图\ref{图4.5}展示了trapezoidal积分方法。trapezoidal方法和Euler方法相似但是再一次在何处选择为函数的导数时出现了不同。像Euler方法一样,通过把前一时刻的解$Y(t)$加到时间步长与平均导数的积上,trapezoidal方法估计$Y(t+1)$的值。trapezoidal算法用了前一时刻$T$解处的导数和当前时刻$T+1$解处的导数。这两个导数值被取了平均,并且在计算中使用了这个平均导数。既不是在前一时间点处评估导数(forward-Euler),也不是在当前时间点处评估导数(backward-Euler),trapezoidal方法平均了两个导数,如图\ref{图4.5}。正因为此,trapezoidal方法无论比forward-还是backward-Euler方法都要表现得更准确。trapezoidal积分公式如方程(\ref{eq:4.15})所示。
\begin{equation}
    Y(t+1)=Y(t)+\frac{timestep*[dY(t)+dY(t+1)]}{2}
    \label{eq:4.15}
\end{equation}

\begin{figure}[htbp]
\small
    \centering
    \includegraphics[width=0.5\textwidth]{figure/Chapter4/图4.5.png}
    \caption{Trapezoidal数值积分公式和该方法的图形表示。}
    \label{图4.5}
\end{figure}

Gear积分方法\cite{Gear}是一种相对比较新的方法。图\ref{图4.6}展示了Gear积分方法。Gear积分方法与传统的积分方法非常不同,用当前解和前两个时刻解的信息来估计时间点$T+1$处的函数值。Gear-2积分公式如公式(\ref{eq:4.16})所示。
\begin{equation}
    Y(t+1)=(\frac{4}{3})*Y(t)-(\frac{1}{3})*Y(t-1)+(\frac{2}{3})*timestep*dY(t+1)
\end{equation}

\begin{figure}[htbp]
\small
    \centering
    \includegraphics[width=0.5\textwidth]{figure/Chapter4/图4.6.png}
    \caption{Gear数值积分公式和该方法的图形表示。}
    \label{图4.6}
\end{figure}

除了Gear-2公式,还有Gear-3,Gear-4,Gear-5,和Gear-6公式。Gear-2公式用了前两个时间点,$Y(t)$和$Y(t-1)$,而Gear-3用了前三个时间点,$Y(t)$,$Y(t-1)$,和$Y(t-2)$。这种规律会一直继续到Gear-6公式。Gear-6公式用了前6个时间点来帮助评估函数在时间$T+1$处的值。

\section{积分方法的精度和稳定性}
由于积分公式的差异,当应用到一个给定的函数上时,每种方法会产生一个不同 的结果。一种给定的积分公式对正确解的预测有多好取决于该方法的精度和稳定性。

在瞬态仿真中,积分方法在每个时间点处近似微分方程的解。因为数值积分的解只是精确解的近似,所以在每个时间点会引入一定量的有限误差。给定时间点的误差的量是该积分方法精度的一种度量。在目前给出的四种积分方法中(backward-Euler,forward-Euler,trapezoidal和Gear),每一种都会在给定的时间点引入不同程度的误差。在数值积分中,每一个时间点引入的误差被称为局部截断误差。图\ref{图4.7}展示了一个任意函数的积分方法的精度和局部截断误差。

\begin{figure}[htbp]
\small
    \centering
    \includegraphics[width=0.5\textwidth]{figure/Chapter4/图4.7.png}
    \caption{数值积分方法的精度表示积分中每个时间点处的误差量。每个时间点处的误差被称为局部截断误差。}
    \label{图4.7}
\end{figure}

数值积分方法的第二条标准是稳定性。在大量的时间点上局部截断误差如何累积是对积分方法稳定性的一种度量。一种稳定的积分方法会在一些时间点上过估计真正的解,而在其他时间点上欠估计真正的解。但是经过大量的时间点(一长段瞬态仿真),稳定的积分方法产生出的结果可以接近近似真实解。相反地,不稳定的方法会在每个时间点上累积局部截断误差。经过大量的时间点,不稳定的积分方法会偏离准确解。图\ref{图4.8}展示了当应用到一个一般函数上时,稳定和不稳定积分方法的结果。

\begin{figure}[htbp]
\small
    \centering
    \includegraphics[width=0.5\textwidth]{figure/Chapter4/图4.8.png}
    \caption{积分方法的稳定性指数值积分解在大量时间点上的表现如何。准确的数值积分方法或许经常并不是稳定的,而稳定的方法或许并不是准确的。}
    \label{图4.8}
\end{figure}

给定一个积分方法,其精度和稳定性由被积分函数的第一和第二阶导数,以及积分过程中使用的时间步长决定。正因为此,没有一种积分方法总是最准确的或者最稳定的。当被应用到一个函数上时,一种积分方法也许会产生大量的局部截断误差,而当被应用到另一个不同的函数上时,会产生很少或者不产生误差。一种给定的积分方法在一些函数上是稳定的,而在另一些上是不稳定的。尽管本书对你在你的电路上应该使用的积分方法会设置一些通用的指导原则,但任何积分方法都容易犯精度和稳定性问题。有知识的仿真用户会学会识别数值积分不精确并采取纠正措施。下面的章节会向你展示如何寻找异常行为,以及做什么来纠正你的仿真。

\section{积分方法的对比}
为了说明不同积分方法之间的区别,这里选择了三个样本时间函数。三个函数都有光滑连续的导数。每个函数的导数都被计算了出来,之后forward-Euler,backward-Euler,trapezoidal和Gear-2积分方法会被应用到计算出的导数上。结果曲线代表原始函数的四种数值积分近似。

方程(\ref{eq:4.17}),(\ref{eq:4.18}),和(\ref{eq:4.19})是用来说明的那三个样本函数。
\begin{equation}
    f(t)=t^2-5t
    \label{eq:4.17}
\end{equation}

\begin{equation}
    f(t)=5-5*exp(-t)
    \label{eq:4.18}
\end{equation}

\begin{equation}
    f(t)=sin(t)
    \label{eq:4.19}
\end{equation}

图\ref{图4.9a},\ref{图4.9b}和\ref{图4.9c}分别表示方程(\ref{eq:4.17})在步长为1s,.5s,和.25s时的情况。通过减小积分步长尺寸,局部截断误差也会降低。虽然没有一种积分方法总是最精确或者最稳定的,但任何积分方法的步长尺寸的降低都会提高解的精度。图\ref{图4.10a},\ref{图4.10b}和\ref{图4.10c}表示方程(\ref{eq:4.18}),而图\ref{图4.11a},\ref{图4.11b}和\ref{图4.11c}表示方程(\ref{eq:4.19})。在每种说明里,减小步长尺寸都会在解的精确度方面产生一个宣称的提高。
\begin{figure}[htbp]
\small
    \centering
    \includegraphics[width=0.5\textwidth]{figure/Chapter4/图4.9a.png}
    \caption{(a)函数$F(t)=t^2-5t$的解析解vs.数值积分解。(重印自Successfully Simulating Circuits With SPICE。授权使用。)}
    \label{图4.9a}
\end{figure}

\begin{figure}[htbp]
\small
    \centering
    \includegraphics[width=0.5\textwidth]{figure/Chapter4/图4.9b.png}
    \caption{(b)函数$F(t)=t^2-5t$的解析解vs.数值积分解。(重印自Successfully Simulating Circuits With SPICE。授权使用。)}
    \label{图4.9b}
\end{figure}

\begin{figure}[htbp]
\small
    \centering
    \includegraphics[width=0.5\textwidth]{figure/Chapter4/图4.9c.png}
    \caption{(c)函数$F(t)=t^2-5t$的解析解vs.数值积分解。(重印自Successfully Simulating Circuits With SPICE。授权使用。)}
    \label{图4.9c}
\end{figure}


\begin{figure}[htbp]
\small
    \centering
    \includegraphics[width=0.5\textwidth]{figure/Chapter4/图4.10a.png}
    \caption{(a)函数$F(t)=5-5e^{-t}$的解析解vs.数值积分解。(重印自Successfully Simulating Circuits With SPICE。授权使用。)}
    \label{图4.10a}
\end{figure}

\begin{figure}[htbp]
\small
    \centering
    \includegraphics[width=0.5\textwidth]{figure/Chapter4/图4.10b.png}
    \caption{(b)函数$F(t)=5-5e^{-t}$的解析解vs.数值积分解。(重印自Successfully Simulating Circuits With SPICE。授权使用。)}
    \label{图4.10b}
\end{figure}

\begin{figure}[htbp]
\small
    \centering
    \includegraphics[width=0.5\textwidth]{figure/Chapter4/图4.10c.png}
    \caption{(c)函数$F(t)=5-5e^{-t}$的解析解vs.数值积分解。(重印自Successfully Simulating Circuits With SPICE。授权使用。)}
    \label{图4.10c}
\end{figure}


\begin{figure}[htbp]
\small
    \centering
    \includegraphics[width=0.5\textwidth]{figure/Chapter4/图4.11a.png}
    \caption{(a)函数$F(t)=sin(t)$的解析解vs.数值积分解。(重印自Successfully Simulating Circuits With SPICE。授权使用。)}
    \label{图4.11a}
\end{figure}

\begin{figure}[htbp]
\small
    \centering
    \includegraphics[width=0.5\textwidth]{figure/Chapter4/图4.11b.png}
    \caption{(b)函数$F(t)=sin(t)$的解析解vs.数值积分解。(重印自Successfully Simulating Circuits With SPICE。授权使用。)}
    \label{图4.11b}
\end{figure}

\begin{figure}[htbp]
\small
    \centering
    \includegraphics[width=0.5\textwidth]{figure/Chapter4/图4.11c.png}
    \caption{(c)函数$F(t)=sin(t)$的解析解vs.数值积分解。(重印自Successfully Simulating Circuits With SPICE。授权使用。)}
    \label{图4.11c}
\end{figure}

之前的声明值得小心考虑。通过减小积分步长尺寸,或者在瞬态仿真中减小时间步长,电容电流和电感电压的数值积分解结果将得到提升。很多SPICE用户知道减小时间步长可以治愈很多瞬态异常。(更精准的数值积分解只是小瞬态时间步长带来的几个提升中的一个。)但是减小时间步长会加剧仿真器的两种不期望得到的特性。减小时间步长强迫仿真器求解更多的点,将会导致更长的仿真运行,而且减小时间步长会增加步入或者靠近模型不连续,进而导致收敛失败。在任何仿真器中,收敛的速度、精度、和稳定性之间都存在着一个精致的平衡。当试着为瞬态分析选择一个保持精度、合理的收敛特性、和合理的仿真时间的时间步长时,这种自然的张力就会被看到。

\section{电子电路的数值积分}
如果我们聚焦在电子电路中最常见的一般类型的函数(线性、分段线性、指数、和正弦),就可以得到关于数值积分的一些通用观察。

第一,forward-Euler方法在这些类型函数上一般是不精确和不稳定的,因此在SPICE中不被使用。第二,剩余的方法——backward-Euler、trapezoidal、和Gear
——可以在一段连续体上被表示。如图\ref{图4.12a},精度在连续体的最左端,而稳定性在连续体的最右端。trapezoidal通常是最精确的;Gear方法,因为它们的光滑或平均的函数,趋向于最稳定。(尽管没有正式的Gear-1方法存在,但backward-Euler方法有时会被称为Gear-1积分方法。)但是太多的稳定性并不总是好的。在高度非线性电路中,前面的解会对未来的解产生极其糟糕的估计。通常,Gear-3、Gear-4、Gear-5、和Gear-6积分方法在Gear-2积分方法上表现出很少或者没有提升,而且,因为这些方法更复杂,包含更多的计算步骤,将导致更长的仿真运行。在仿真中,Gear-3,Gear-4,Gear-5,和Gear-6方法都不推荐使用。对于一般的电子电路仿真,backward-Euler、trapezoidal、和Gear-2积分方法会产生出最好的结果。这三种方法如图\ref{图4.12b}所示。

\begin{figure}[htbp]
\small
    \centering
    \includegraphics[width=0.5\textwidth]{figure/Chapter4/图4.12a.png}
    \caption{(a)精度和稳定性之间的连续变化。}
    \label{图4.12a}
\end{figure}

\begin{figure}[htbp]
\small
    \centering
    \includegraphics[width=0.5\textwidth]{figure/Chapter4/图4.12b.png}
    \caption{(b)SPICE仿真中建议的积分方法。}
    \label{图4.12b}
\end{figure}

在我们继续前进之前,强调方程(\ref{eq:4.10a})和(\ref{eq:4.10b})之间的差异是重要的。尽管这两个方程本质上是一样的,a是积分的方程,而b是微分的方程。积分趋向于光滑或者平均的函数,而微分趋向于有噪音的和有时不可靠的函数。尽管知道是数值积分,但SPICE反转了积分公式中的相等,用数值差分来决定电容电流和电感电压。

阶跃响应是系统稳定性的一种度量。通过给数值积分函数施加一个阶跃函数,可以做出几项非常重要的观察。在图\ref{图4.13}中,给一个1-farad的电容加了一个斜坡电压。电压每秒恒定增加1 volt。在时间$t=1$处,电压保持恒定。用解析积分,电容电流可以直接从方程\ref{eq:4.1}计算得到。电容电压和电流如图\ref{图4.13}所示。

\begin{figure}[htbp]
\small
    \centering
    \includegraphics[width=0.5\textwidth]{figure/Chapter4/图4.13.png}
    \caption{电容电压和计算得到的电流响应。(重印自Successfully Simulating Circuits With SPICE。授权使用。)}
    \label{图4.13}
\end{figure}

通过用数值积分取代解析积分,我们可以比较当被用作数值微分时每种积分方法执行的怎么样。

图\ref{图4.14a}和\ref{图4.14b}展示了在backward-Euler积分方法下的电容电流响应。在图\ref{图4.14a}中,backward-Euler积分方法的固定步长尺寸为.025S。图\ref{图4.14b}是同样的计算结果,只不过固定步长尺寸为.0025s。越小的步长尺寸虽然提高了积分结果,但是要花费更长的时间来计算。
\begin{figure}[htbp]
\small
    \centering
    \includegraphics[width=0.5\textwidth]{figure/Chapter4/图4.14a.png}
    \caption{(a)从backward-Euler积分计算得到的电容电压和电流响应。步长为.025S。(重印自Successfully Simulating Circuits With SPICE。授权使用。)}
    \label{图4.14a}
\end{figure}

\begin{figure}[htbp]
\small
    \centering
    \includegraphics[width=0.5\textwidth]{figure/Chapter4/图4.14b.png}
    \caption{(b)从backward-Euler积分计算得到的电容电压和电流响应。步长为.0025S。(重印自Successfully Simulating Circuits With SPICE。授权使用。)}
    \label{图4.14b}
\end{figure}

图\ref{图4.15a}和\ref{图4.15b}展示了使用trapezoidal积分方法时产生的电容电流响应。在图\ref{图4.15a}中,电容电流在时间$t=1$后开始围绕正确值0震荡。虽然这个结果让很多读者感到惊奇,但是ringing效应是一种著名的trapezoidal失效机制。在很多含有突变或者不连续导数的函数中,$dV/dt$,使用大步长的trapezoidal积分方法会导致出现被称为trapezoidal震荡的现象。因为电路特征值\cite{Leon}和使用的时间步长尺寸,所以发生了震荡。虽然震荡不经常发生但是当电路特征值比较小和时间步长比较大时,就会发生震荡。Trapezoidal震荡不是电路中元件的结果;震荡只是数值积分算法的一种简单的错误结果。Trapezoidal震荡是一种准备好的可观察的信号,提示数值积分方法预测正确的响应失败,需要采取纠正措施。
\begin{figure}[htbp]
\small
    \centering
    \includegraphics[width=0.5\textwidth]{figure/Chapter4/图4.15a.png}
    \caption{(a)从trapezoidal积分计算得到的电容电压和电流响应。步长为.025S。(重印自Successfully Simulating Circuits With SPICE。授权使用。)}
    \label{图4.15a}
\end{figure}

在SPICE中可行的三种积分方法中,trapezoidal方法是唯一一个能够表现出震荡行为的。对大多数函数,通过减小积分的步长尺寸,震荡可以被压制或者消除。通过把步长尺寸减小到.0025S,图\ref{图4.15b}中看到的震荡比之前看到的要明显地减弱了。进一步减小步长尺寸会消除震荡。

\begin{figure}[htbp]
\small
    \centering
    \includegraphics[width=0.5\textwidth]{figure/Chapter4/图4.15b.png}
    \caption{(b)从trapezoidal积分计算得到的电容电压和电流响应。步长为.0025S。(重印自Successfully Simulating Circuits With SPICE。授权使用。)}
    \label{图4.15b}
\end{figure}

SPICE用一种动态时间步长控制算法和源断点调整(这些会在第\ref{chap:5}中详细介绍)来帮助最小化trapezoidal震荡。对很多仿真来说,这些技术在控制震荡方面工作地很好,但是对一些电路,trapezoidal震荡还能早就被观察到。

图\ref{图4.16a}和\ref{图4.16b}展示了在Gear-2积分方法下产生的电容电流响应。在图\ref{图4.16a}中,时间$t=1$之后的电容电流表现出对正确值0的过冲。这种过冲是Gear积分方法一种著名的特性。因为Gear积分方法使用之前的时间点来帮助确定下一个解,所以带突变的电路函数经常表现出对正确值过冲的结果。就像trapezoidal震荡一样,Gear引起的过冲并不是电路中元件的结果;过冲是积分算法引入的一种简单的误差。Gear-2积分方法是SPICE中最稳定的一种积分方法。然而,就是这种给Gear方法增加了稳定性的相同的机制表现出对突变的抗拒。这种抗拒变化产生的过冲如图\ref{图4.16a}所示。
\begin{figure}[htbp]
\small
    \centering
    \includegraphics[width=0.5\textwidth]{figure/Chapter4/图4.16a.png}
    \caption{(a)从Gear-2积分计算得到的电容电压和电流响应。步长为.025S。(重印自Successfully Simulating Circuits With SPICE。授权使用。)}
    \label{图4.16a}
\end{figure}

一种减小或者消除过冲的方法是再次在积分中减小使用的时间步长。图\ref{图4.16b}展示了在时间$t=1$处,一个多小的步长尺寸是如何减小了过冲的。
\begin{figure}[htbp]
\small
    \centering
    \includegraphics[width=0.5\textwidth]{figure/Chapter4/图4.16b.png}
    \caption{(b)从Gear-2积分计算得到的电容电压和电流响应。步长为.0025S。(重印自Successfully Simulating Circuits With SPICE。授权使用。)}
    \label{图4.16b}
\end{figure}

在前面展示中看到的Trapezoidal震荡和Gear过冲是也许会给你的瞬态仿真增添不精确的两种基本失效机制。这些同样的特征是一种用户需要在瞬态仿真中查看的提示。SPICE不会把这些作为异常行为标出。作为SPICE用户,你必须观察这些行为,并对你的仿真做出合适的调整。

\subsection{选择一种积分方法}
因为SPICE对整个电路只使用一种积分方法,所以仿真用户需要为他们的电路选择最好的方法。SPICE的作者们原始选择把trapezoidal积分作为默认的积分方法。trapezoidal方法作为默认方法是一项好选择,因为在电子电路中最常看到的函数类型中,trapezoidal趋向于产生最精确的结果。除了算法的精确度,trapezoidal方法也比Gear积分方法估计得更快。对很多类型的电路,trapezoidal积分方法对于仿真是最好的选择。

但是trapezoidal积分方法并不总是最好的选择。除了trapezoidal积分方法,SPICE的作者们也实现了backward-Euler和Gear积分方法,并且给仿真用户选择具体使用方法的选项。

\subsection{改变积分方法}
SPICE用户或许会选择trapezoidal,backward-Euler,或者Gear积分方法。改变积分方法的开关在.OPTIONS声明中可以被容易地找到。选择Gear-2或者trapezoidal积分方法,使用下面的选项:

.OPTIONS METHOD=TRAP (默认)

.OPTIONS METHOD=GEAR

如果选项METHOD=GEAR被选中,SPICE会使用Gear-2积分方法。尽管本书没有推荐使用更高阶的Gear方法(阶数高于2),但高阶的Gear方法可以通过MAXORD选项来选择,如下所示:

.OPTIONS METHOD=GEAR MAXORD=3

.OPTIONS METHOD=GEAR MAXORD=4

.OPTIONS METHOD=GEAR MAXORD=5

.OPTIONS METHOD=GEAR MAXORD=6。

选择backward-Euler积分方法,用MU选项:
.OPTIONS MU=0

backward-Euler积分方法是在Nagel\cite{Nagel}完成了他在程序上的工作之后的版本中被加入到SPICE中的。正是这个原因,在原始的SPICE2用户指南\cite{SPICE2G.6}中,MU选项从来没有被说明。因此,很多仿真用户不关心MU选项。

在SPICE中,原始的trapezoidal算法已经被修改得把标量MU加进了形式中。因为这两种算法公式的相似性,如果MU的值是0.5,公式会被降为简单的trapezoidal积分方法。如果MU被设置为0,公式会被降为backward-Euler积分方法。而如果MU被设置为0到0.5之间的一个数,两种方法的加权平均值会被使用。在SPICE中,MU的值被默认设置为0.5。设置选项MU=0会变化到backward-Euler积分,而设置在0和0.5之间会结合那两种积分方法。

\section{检测和更正积分失效}
SPICE中的三种积分方法没有一种对所有的瞬态电路都能产生准确的结果。三种积分方法都包含着不同的失效机制。偶尔,在瞬态分析中,一定的电压和电流波形会激发出一个或者多个积分失效机制。激发出这些失效机制中的一个就会在仿真输出里引入误差。误差的量从毫伏和微微安培变化到几十伏和安培。这取决于仿真用户检测数值积分错误并采取纠正措施。SPICE没有检测数值积分失效的内在机制,因此不会标定出有误差的结果。数值积分失效必须被用户在仿真输出中检测。只有那时才可以采取纠正措施。

\subsection{梯形积分失效}
trapezoidal积分方法是SPICE中默认使用的方法。trapezoidal方法结合了好的精度和低的计算要求。对很多电路,trapezoidal积分方法对仿真是最好的选择。

什么时候trapezoidal方法也不是一个好的选择呢?那就是trapezoidal方法开始给仿真结果引入明显的误差的时候。trapezoidal积分方法有两种著名的失效机制用户应该学会识别。一个失效机制是trapezoidal震荡;你在图\ref{图4.15a}的例子中已经看过。另一个失效机制是累计误差。一旦知道这些失效机制存在,两种机制都可以从仿真输出的特征中被识别出来。如果你知道在输出中查看什么,这些失效机制的特征是容易被识别的。通过学习失效机制的特征,SPICE用户会学到识别trapezoidal积分失效的存在,并能便捷地切换到一个更合适的积分方法。

\subsubsection{Trapezoidal震荡}
Trapezoidal震荡是一种著名的,有很多资料的trapezoidal积分方法\cite{Leon}。当积分步长太大以至于无法跟上给定函数的曲率时,Trapezoidal震荡就会发生。这种失效机制的结果是一种可预测的解,表现为围绕正确的解震荡。硬盘文件ch4-17a.cir和ch4-17b.cir包含了一个MOS电容的瞬态仿真的网表。图\ref{图4.17a}展示了电路ch4-17a.cir的仿真结果。图\ref{图4.17a}的仿真结果表现出围绕被图\ref{图4.17b}预测的正确值震荡的现象。在仿真电路ch4-17a.cir时,trapezoidal震荡给输出引入了大量的误差。用下面的命令仿真电路:

SIM CH4-17A.CIR

你的结果应该和图\ref{图4.17a}匹配。
\begin{figure}[htbp]
\small
    \centering
    \includegraphics[width=0.5\textwidth]{figure/Chapter4/图4.17a.png}
    \caption{(a)MOS电容的瞬态仿真。(重印自Successfully Simulating Circuits With SPICE。授权使用。)}
    \label{图4.17a}
\end{figure}

\begin{figure}[htbp]
\small
    \centering
    \includegraphics[width=0.5\textwidth]{figure/Chapter4/图4.17b.png}
    \caption{(b)MOS电容纠正过的瞬态仿真。(重印自Successfully Simulating Circuits With SPICE。授权使用。)}
    \label{图4.17b}
\end{figure}

为了从仿真结果中消除trapezoidal震荡,SPICE用户可以切换到Gear或者backward-Euler积分。Gear和backward-Euler积分都不会遭遇震荡异常。图\ref{图4.17b}展示了电路ch4-17b.cir的仿真结果。ch4-17b.cir电路网表除了增加选项声明调用backward-Euler积分方法(.OPTIONS MU=0)外,其他方面与ch4-17a.cir完全一致。backward-Euler方法彻底消除了震荡。

用下面的命令仿真电路:

SIM CH4-17B.CIR

你的结果应该和图\ref{图4.17b}匹配。

除了增加选项声明调用了Gear-2积分方法(.OPTIONS METHOD=GEAR)外,ch4-17c.cir电路网表与ch4-17a.cir完全一致。

ch4-17c.cir的仿真结果如图\ref{图4.17c}所示。图\ref{图4.17c}展示了ch4-17c.cir的仿真结果,并且在两个转换点处都阐明了明显的过冲。在ch4-17c.cir的仿真中,和很多转换电路中,Gear积分方法,因为它们的平均效应,在电路电压和电流突然变化的地方会倾向于产生过冲。像trapezoidal积分引起的震荡,在真实电路中不会存在过冲。过冲仅仅是Gear积分方法引入的一个误差。
\begin{figure}[htbp]
\small
    \centering
    \includegraphics[width=0.5\textwidth]{figure/Chapter4/图4.17c.png}
    \caption{(c)用Gear-2进行瞬态仿真。(重印自Successfully Simulating Circuits With SPICE。授权使用。)}
    \label{图4.17c}
\end{figure}

虽然Gear或者backward-Euler积分方法都可以被选择来消除trapezoida震荡,但是对于这个电路,backward-Euler方法是最好的选择。在比较了全部三种仿真的仿真结果,并且知道了仿真结果应该是什么样后,backward-Euler积分方法对这个仿真是最好的选择。

几种类SPICE仿真器不容许用户切换Gear或者backward-Euler积分。那么这些工具的用户为了纠正trapezoidal震荡问题可以做什么呢?答案是减小最大瞬态时间步长。(注意:第\ref{chap:5}章会详细介绍如何限制最大瞬态时间步长。)图\ref{图4.14b},\ref{图4.15b},和\ref{图4.16b}论证了如何减小瞬态步长尺寸同时消除,或者减少,trapezoidal震荡和Gear过冲。在SPICE中,减小最大瞬态步长尺寸是另一条减少误差和提高所有积分方法精度的途径。

但是减小时间步长并不总是希望的。在第\ref{chap:non-conver}章,已经展示过减小时间步长会使得仿真器运行得更久,并且使仿真更倾向于不收敛问题。无论何时,更换积分方法都是纠正数值积分问题最受欢迎的途径。

Trapezoidal震荡是由数值积分方法失效引起,并不是来自电路描述或者器件模型的不合适。有时候,如图\ref{图4.17a}所示,trapezoidal震荡会很大并且从根本上不合实际;另外一些时候,震荡会非常微弱。并不是所有的trapezoidal震荡都会像前面的例子中宣称的那样。图\ref{图4.18a}展示了电路文件ch4-18a.cir的仿真结果。仿真输出中的微小波纹是trapezoidal震荡的结果。图\ref{图4.18b}展示了用Gear积分方法进行相同仿真的结果。

\begin{figure}[htbp]
\small
    \centering
    \includegraphics[width=0.5\textwidth]{figure/Chapter4/图4.18a.png}
    \caption{(a)带trapezoidal震荡的瞬态仿真。(重印自Successfully Simulating Circuits With SPICE。授权使用。)}
    \label{图4.18a}
\end{figure}

\begin{figure}[htbp]
\small
    \centering
    \includegraphics[width=0.5\textwidth]{figure/Chapter4/图4.18b.png}
    \caption{(b)纠正过的瞬态仿真。(重印自Successfully Simulating Circuits With SPICE。授权使用。)}
    \label{图4.18b}
\end{figure}

用下面的命令仿真这些电路:

SIM CH4-18A.CIR

之后

SIM CH4-18B.CIR

你的结果应该与图\ref{图4.18a}和\ref{图4.18b}匹配。

因为SPICE没有检测trapezoidal震荡的算法,所以程序不会警告用户积分方法失效。Trapezoidal震荡是一种必须被用户观察的现象。寻找的线索是无法解释的或者未被预料到的震荡,特别是当震荡围绕着本应该正确的结果。这自然引出了一个问题,你如何才能知道仿真输出中的震荡是真实的或者仅仅是trapezoidal算法的失效。区分这两种最好的办法是用Gear或者backward-Euler积分方法重新运行仿真。选择了新积分方法,如果是由trapezoidal算法引起的失效,震荡会消失。如果电路响应本来就是震荡的起因,那么所有三种积分方法——trapezoidal,Gear,和backward-Euler——会重新产生震荡行为。
\subsubsection{累积误差}
累积误差是trapezoidal积分方法中的第二种失效机制。累积误差经常发生在周期性电路和长时间瞬态仿真中。这种失效机制的结果通常是一种无法解释的电路行为。

一个典型的瞬态仿真会要求进行几千次时间点计算。在每一个时间点,数值积分算法会在解中引入一定量的误差。在一些时间点,误差会高估准确值,而在另一些时间点,误差会低估准确值。因为之前解中的误差会被带到下一个解中,高估/低估的误差通常在大量的时间点中会彼此抵消。但是有时候累积误差会随着每个新解增加。当这个发生时,仿真结果将偏离准确值。经常,仿真输出表现出完全的不精确。在很多例子中,有误差的输出会表现得违反物理规律\cite{chap4-6}(或者电子规律)。

电路文件ch4-19a.cir是一个555计时器的瞬态行为的列表。在ch4-19a.cir网表中,一个外接电容和几个外接电阻与计时器相连。在这种构造中,计时器表现得像一个异步振荡器。仿真结果如图\ref{图4.19a}所示。在这个仿真中,累积误差引入了一种数学上的稳定状态,但这个情况不会在实际的555计时器中出现。因为电路是一个异步振荡器,一旦积分误差累积到电路稳定的点,输出会在剩余仿真的状态里保持。用下面的命令仿真该电路:

SIM CH4-19A.CIR

你的结果应该和图\ref{图4.19a}匹配。

\begin{figure}[htbp]
\small
    \centering
    \includegraphics[width=0.5\textwidth]{figure/Chapter4/图4.19a.png}
    \caption{(a)累积误差仿真失效。(重印自Successfully Simulating Circuits With SPICE。授权使用。)}
    \label{图4.19a}
\end{figure}

电路文件ch4-19b.cir除了另加选项声明调用Gear-2积分方法外,其余与555计时器完全一样。这个仿真的结果如图\ref{图4.19b}所示。用下面的命令仿真该电路:

SIM CH4-19B.CIR

你的结果应该和图\ref{图4.19b}匹配。

\begin{figure}[htbp]
\small
    \centering
    \includegraphics[width=0.5\textwidth]{figure/Chapter4/图4.19b.png}
    \caption{(b)纠正的瞬态仿真。(重印自Successfully Simulating Circuits With SPICE。授权使用。)}
    \label{图4.19b}
\end{figure}

trapezoidal积分中的累积误差通常是以一种无法解释的仿真行为为特征。在这些例子中,切换到更稳定的积分方法(Gear-2)或者减小最大瞬态时间步长可以最小化累积误差,并能恢复正常的电路行为。(作者提醒:经验证明累积误差并不是不可解释的电路行为的唯一起因!)

\subsection{后向欧拉积分失效}
尽管backward-Euler方法不会经受trapezoidal方法的数值震荡,但当被应用到非线性波形时,它也许会遭遇更多的局部截断误差。对正弦和指数型波形,backward-Euler积分方法会产生大量的误差,这些误差将导致累积误差失效和不可预测的仿真结果。对这些类型的电路,backward-Euler积分方法是不被推荐的。backward-Euler最好被用在线性和分段线性的波形中。backward-Euler方法对数字电路来说是一种好的选择。

电路文件ch4-20a.cir是一个理想谐振器的瞬态行为列表。电路由一个电容和一个电感构成。瞬态仿真从一个流过电感的初始电流开始被启动。因为两个器件都是理想的(没有寄生泄露),电路在整个瞬态仿真期间应该持续震荡。图\ref{图4.20a}展示了使用backward-Euler积分方法的仿真结果。对于正弦和非线性波形,backward-Euler方法会在仿真中引入很多局部截断误差。在这些电路中,局部截断误差会快速积累,导致累积误差失效。在仿真ch4-20a.cir中,误差很快开始阻尼理想振荡。用下面的命令仿真这个电路:

SIM CH4-20A.CIR

你的结果应该和图\ref{图4.20a}匹配。
\begin{figure}[htbp]
\small
    \centering
    \includegraphics[width=0.5\textwidth]{figure/Chapter4/图4.20a.png}
    \caption{(a)累积误差仿真失效。(重印自Successfully Simulating Circuits With SPICE。授权使用。)}
    \label{图4.20a}
\end{figure}

电路文件ch4-20b.cir是同样的一个振荡器,除了选项重置仿真器为trapezoidal积分方法。仿真这个电路用下面的命令:

SIM CH4-20B.CIR

你的结果应该和图\ref{图4.20b}匹配。图\ref{图4.20b}展示了合适的仿真输出。
\begin{figure}[htbp]
\small
    \centering
    \includegraphics[width=0.5\textwidth]{figure/Chapter4/图4.20b.png}
    \caption{(b)纠正的瞬态仿真。(重印自Successfully Simulating Circuits With SPICE。授权使用。)}
    \label{图4.20b}
\end{figure}

像trepezoidal累积误差失效一样,backward-Euler积分中的累积误差以无法预期或者无法解释的仿真结果为特征。当仿真非线性或者正弦波形时,Backward-Euler数值积分特别容易倾向于累积误差失效,所以最好被用在线性或分段线性波形中。

\subsection{Gear积分失效}
像backward-Euler积分,Gear积分不会遭遇trapezoidal震荡,但是Gear积分会遭受大量的局部截断误差,特别在包含高度非线性或者开关波形的电路中。

Gear积分方法使用了一种平均的技术来预测下一个解。在理论上,这种平均的技术应该使得算法更加稳定(在大量的时间点上更少的累积误差)。但是,因为Gear方法使用过去的时间点的值来帮助预测应该在哪里找到下一个解,所以对于转换波形和分段线性波形,Gear方法会过冲正确的解。(你应该在图\ref{图4.16a}和\ref{图4.17c}中已经看到过这种现象了。)在开关电路中所谓的过冲是SPICE中Gear积分方法的一种特征性失效。

很多开关电路在转换点会表现出过冲。又一次,仿真用户要面对在他们的仿真中看到的过冲是真实的还是仅仅是简单的一个Gear积分误差。确定过冲真实起因的最好的办法是切换积分方法。无论是trapezoidal还是backward-Euler积分方法,都不会产生Gear积分方法那样的过冲。用其他积分方法中的一种重新运行仿真。如果过冲是由电路的机理引起,仿真结果应该和Gear积分结果一致。如果过冲是由Gear引入的误差引起,过冲会消失。

\section{对积分设置的建议}
当仿真瞬态电路时,用户必须学会查看数值积分异常。表\ref{表4.2}是不同类型电路的建议积分方法列表。表\ref{表4.3}是数值积分失效机制和对应的纠正措施的总结。在表\ref{表4.3}中,纠正措施的列表包含了减小最大瞬态时间步长。设置最大时间步长是下一章的主题。在用该技术纠正数值积分问题之前先读第\ref{chap:5}章。如果它们确实发生了,用这些表来最小化数值积分失效,并纠正积分失效。

\begin{figure}[htbp]
\small
    \centering
    \includegraphics[width=0.5\textwidth]{figure/Chapter4/表4.2.png}
    \caption{建议的积分方法}
    \label{表4.2}
\end{figure}

\begin{figure}[htbp]
\small
    \centering
    \includegraphics[width=0.5\textwidth]{figure/Chapter4/表4.3.png}
    \caption{常见的数值积分失效}
    \label{表4.3}
\end{figure}

\section{其他仿真器中的数值积分}
\subsection{Hspice}
Meta-Software的HSpice包含标准SPICE的所有三种积分方法,但是当Hspice用户仿真ch4-17a.cir和ch4-18a.cir时,他们不会看见在标准SPICE输出中看到的异常行为。对这个的解释不得不对Hspice如何执行瞬态分析进行更多的讨论,而不是对积分方法是如何被施加的进行探讨\cite{chap4-7}。

标准SPICE和这里回顾的所有其他仿真器,默认地,使用了一种叫作局部截断误差时间步长控制算法的时间步长控制算法。Meta-Software已经为瞬态分析开发了一种受知识产权保护的时间步长控制算法,被称为DVDT。DVDT算法在几个标准时间步长控制算法的缺陷上进行了提升,并且结果是倾向于遭遇更少的数值积分问题。(记住步长尺寸和数值积分精度之间的关系。)

Hspice没有遭遇同样问题的第二个原因是很多I-V和C-V器件方程已经被重写,并且方程(不连续)的许多尖利的边已经被平滑。因为这些尖利的边激发trapezoidal震荡,不连续上的减少可以降低trapezoidal震荡的发生率。(Hspice用户会注意到电路ch4-17a.cir的Hspice仿真是如何产生出一个看起来非常不同的输出。)

\subsection{IS\_{Spice}}
Intusoft的IS\_{Spice}是直接基于SPICE 2G.6的。IS\_{Spice}提供给用户对trapezoidal,backward-Euler,和Gear积分方法的标准选择。Trapezoidal积分是IS\_{Spice}中的默认方法。

\subsection{Micro-Cap IV}
Spectrum Software的Micro-Cap IV也是基于SPICE2G.6,但是完全被用C重写了。Micro-Cap IV不支持backward-Euler或者Gear积分方法\cite{chap4-9}。如果(trapezoidal)积分问题确实出现,用户必须减小程序的最大时间步长。Micro-Cap IV用户应该仿真ch4-17a.cir和ch4-18a.cir来论证数值积分失效的影响。之后在减小最大时间步长的过程中重新仿真电路,直到误差消失。

\subsection{Pspice}
MicroSim的Pspice要不使用trapezoidal,要不使用Gear-2数值积分方法。但是不像标准的SPICE,Pspice不给用户提供选项来手动选择积分方法。默认地,Pspice用trapezoidal数值积分来开始瞬态仿真。随着仿真进行,Pspice中的积分函数查看哪个数值条件会诱导出trapezoidal震荡。如果trapezoidal震荡的条件发生,Pspice在震荡开始前自动改变到Gear-2积分。

\section{总结}
SPICE中的任一积分方法在瞬态分析中的每一个时间点上都会产生一定量的误差。在很多仿真中,误差是小的。在这些例子中,仿真结果的整体精度会在程序的误差容限内。但是当积分不精确变得很大时,仿真结果会偏离正确解。相比于原始信号,偏离的量也许会无穷小或者大几个数量级!当分析仿真结果时,SPICE用户必须学会识别数值积分误差。SPICE没有检测和警告用户数值积分失效的算法。对你的输出保持警惕和质疑。仿真产生了你期望的结果吗?如果没有,检查仿真。质疑非预期的行为。对积分引起的失效保持警惕和知道如何修正这些失效是一个真正的仿真工匠的水印。

前面的章节描述了SPICE数值积分方法最可能产生的误差类型。这些是SPICE用户必须学会查看的有问题的行为类型。SPICE用户在对输出应该是什么没有一些合理的期望时,从不应该运行仿真。声明是重要的,因为没有对输出的合理期待,检测数值积分失效是不可能的。

因为SPICE和几乎所有的类SPICE仿真器都把trapezoidal积分作为默认的积分方法,所以你应该首先开始查看trapezoidal失效。与trapezoidal积分相关的两种积分失效是trapezoidal震荡和累积误差。在这两种失效机制中,trapezoidal震荡是最常见的。图\ref{图4.17a}和\ref{图4.18a}是对trapezoidal震荡的展示。用户也应该查看累积积分误差的信号。图\ref{图4.19a}和\ref{图4.20a}是对累积误差失效的展示。

当你切换到Gear积分方法,在开关电路中观察不期望的过冲和累积误差。当你切换到backward-Euler积分方法,观察累积误差失效。

在SPICE中,在瞬态仿真过程中,数值积分算法照常产生极精确的,高度可靠的结果。但是偶尔,数值积分函数会产生错误性结果。因为SPICE不自动检测数值积分误差,所以检测和纠正这些误差的责任自然落在了用户身上。如果用户知道查看它们,大多数数值积分失效可以被识别出来,而所有的数值积分失效都可以被修正。纠正积分失效会涉及切换算法或者减小最大内部时间步长。成为一名有经验的SPICE用户需要太多,不光是简单地学会创造输入网表。它也包括知晓SPICE的强项和弱项,以及知晓如何补偿程序的弱势方面。

\bibliographystyle{plain}
\bibliography{Reference/chapter4-reference}