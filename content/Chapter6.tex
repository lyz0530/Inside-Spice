\chapter{SPICE选项}
SPICE输入文件中最重要的命令之一是.OPTIONS声明。.OPTIONS声明被用来设置控制仿真的参数值。参数调整SPICE的数值算法,改变仿真结果出现在输出文件中的形式,产生关于仿真或者网表的额外信息,当然也包括一系列不同的函数。

理解.OPTIONS声明参数对产生快速,准确,收敛的SPICE仿真至关重要。虽然默认的参数设置在一些电路上产生精确的结果,但是很多其他的电路会要求对一个或者多个参数调整。记着SPICE原本是用来优化集成电路级的电压和电流。板级和系统级的设计人员以及电源供应设计人员已经见过几个参数值,这些参数值因为流过它们的电路的电压和电流的幅度,必须被简单地改变。表现出不收敛行为的电路可以通过调整控制SPICE数值算法的参数,被强迫到收敛状态。表现出时间步长控制异常或者数值积分无力的瞬态仿真可以通过施加合适的参数设置来纠正。第\ref{chap:non-conver}、\ref{chap:4}和\ref{chap:5}章都集中在产生更快、更准和可收敛的仿真。在这些章中,特定的问题通过施加一个或者多个.OPTIONS参数被纠正了。.OPTIONS声明作为一条能够同时纠正电路仿真问题和优化仿真运行时间的大道。对快速、准确和可收敛的仿真,理解.OPTIONS声明不只是希望的,而且是必须的。

SPICE有34个用户选择选项。大多数供应商提供的基于SPICE的仿真器使用同样的选项,而且经常加入很多新的选项。在34个可行的选项中,33个被Berkekey SPICE2G.6用户指南记录\cite{chap6-1}。一个从来没有被加入原始文档。在这34个选项中,6个作为标志工作,并且当选项在.OPTIONS行命名的时候发出事件信号。举个例子,为了打印你的仿真运行的会计信息(运行时间、使用的迭代次数,等等),用户会指明.OPTIONS ACCT。其余的29个选项作为参数工作,而且要求在.OPTIONS行配置一个数值值。例如,为了设置直流偏置点计算容许的迭代次数,用户会指定.OPTIONS ITL1=X,这里X是容许的迭代次数。

.OPTIONS声明合适的命令语法是:

.OPTIONS <option1 option2 option3 ...>

然而,因为SPICE处理输入文件行的方式,只要求.OPTIONS命令中的前三个字符。下面所有的都是.OPTIONS声明接收的形式。

.OPTIONS <option1 option2 option3 ...>

.OPTION <option1 option2 option3 ...>

.OPTIO <option1 option2 option3 ...>

.OPTI <option1 option2 option3 ...>

.OPT <option1 option2 option3 ...>

在一个输入文件中,用户会指定不止一个.OPTIONS行。如果同样的选项被声明了不止一次,之后选项的最后一个值会被使用。这在含有不同模块或者子节,每个中可能含有不同选项声明的仿真中是重要的。SPICE不容许网表的不同子节带着特定的选项设置工作。一旦设定,选项就会被全局应用到电路,直到.OPTIONS声明重置该值。

所有选项的名称都包含2到6个字母数字字符。一些名称被选择地很好,清楚地表达了选项的含义。例如,选择仿真中使用的数值积分方法的选项, METHOD='GEAR',是容易理解和使用的。但是经常,名称无法自我解释。例如,在仿真中使用的时间步长控制算法的选项是LVLTIM=X。最后,有很少一部分选项的名称表现出来是显式的和有意义的,但是并不按照名称暗示的那样工作。例如,RELTOL,ABSTOL,和VNTOL都决定了仿真结果的精度,因为或许从名称的TOLerance部分可以推断。选项CHGTOL和TRTOL都是局部截断误差时间步长控制算法中一个缺点的纠正因子。它们对仿真结果的精度没有直接影响,即使它们在它们的名称中都有TOL。对每个选项的清楚理解可帮助建立一个高质量仿真的坚实基础。

作为一个初始SPICE用户,作者在SPICE用户指南\cite{chap6-1}中回忆尝试解码神秘的每个选项的一行描述。因为这个年轻的工程师对很多选项做什么几乎没有概念,很多行为异常的仿真经常被用一系列不同的选项随意浇灭。对每个仿真,这个年轻的工程师会焦急地等待结果,每次都希望这些选项设置会是成功的钥匙。经常,在跌跌撞撞到正确的组合前,几个小时的仿真时间会被消耗。太多次,正确的组合从来不会被找到。

\section{SPICE选项定义}
在这节中,34个标准SPICE选项中的每一个会被定义地清清楚楚。只要哪里施加,展示每个选项值对特定的电路应该被如何设置的公式就会给出。一些选项不管仿真的类型应该总是被设置。其他的只会偶尔地需要。对每一个选项,关于是不是应该总被设置和/或什么时候应该被设置的推荐都会给出。

对使用类SPICE仿真器中的一个的设计师,这里详细描述的大部分选项与你的仿真器中的相同名称的选项有着一一对应的关系。附录\ref{appendix-A}列出了几个受欢迎的电路仿真器的.OPTIONS参数,并在有可能的地方,展示了SPICE2G.6选项和供应商提供的选项之间的对应关系。

前8个选项会被分类为控制输出文件格式的选项,接下来的18个控制SPICE的数值算法,4个决定了默认的MOS门几何,而最后,多于4个只被分类为杂项选项。选项会按照这个顺序被提出。

\section{输出文件格式选项}
输出文件格式选项包括ACCT,LIST,NOMOD,NOPAGE,NODE,OPTS,LIMPTS=X,和NUMDGT=X。

\subsection{.OPTIONS ACCT}
该记录选项是一个标识,指导SPICE打印关于刚才仿真任务的47个运行时间数据。结果会出现在输出文件中。输出文件的记录部分含有丰富的仿真运行信息。在记录部分,用户会找到关于仿真速度和效率的信息,每种分析类型消耗的时间和迭代次数,电路中元件的尺寸和数量,解矩阵的尺寸,分析点的数量,被仿真的电路元件或节点的最大数目。

图\ref{图6.1}展示了一个输出文件中记录信息的例子。信息被分段分为两个不同的形式,仿真元件附属的数值和运行时间数据。

\begin{figure}[htbp]
\small
    \centering
    \includegraphics[width=0.5\textwidth]{figure/Chapter6/图6.1.png}
    \caption{ACCT选项输出。}
    \label{图6.1}
\end{figure}

记录部分值的第一行与电路中的元件相关。第二行显示关于分析的信息。第三行揭示了系统方程的细节,而最后一行包含了关于瞬态分析的数据和仿真要求的内存。

描述了电路元件,分析,系统方程,和瞬态以及内存项的标题展示在表\ref{表6.1a},\ref{表6.1b},\ref{表6.1c},和\ref{表6.1d}。用斜体标记的项对仿真用户是最有用的。

紧接着四行数据之后是四列显示了每个程序片段使用的运行时间数据和迭代次数。程序片段标题显示在表\ref{表6.2}。再一次,用斜体显示的标题对大多数用户是有用的。

在记录部分发现的信息里最好用途中的两个是提高仿真的速度或者效率,并且决定你的仿真器支持的最大电路尺寸或者最大分析时间。

\begin{figure}[htbp]
\small
    \centering
    \includegraphics[width=0.5\textwidth]{figure/Chapter6/表6.1a.png}
    \caption{电路元件信息}
    \label{表6.1a}
\end{figure}

\begin{figure}[htbp]
\small
    \centering
    \includegraphics[width=0.5\textwidth]{figure/Chapter6/表6.1b.png}
    \caption{分析信息}
    \label{表6.1b}
\end{figure}

\begin{figure}[htbp]
\small
    \centering
    \includegraphics[width=0.5\textwidth]{figure/Chapter6/表6.1c.png}
    \caption{电路元件信息}
    \label{表6.1c}
\end{figure}

\begin{figure}[htbp]
\small
    \centering
    \includegraphics[width=0.5\textwidth]{figure/Chapter6/表6.1d.png}
    \caption{瞬态分析和内存信息}
    \label{表6.1d}
\end{figure}

\begin{figure}[htbp]
\small
    \centering
    \includegraphics[width=0.5\textwidth]{figure/Chapter6/表6.2.png}
    \caption{程序片段标题}
    \label{表6.2}
\end{figure}

\subsubsection{使用记录信息来提高仿真}
为了提高仿真速度,观察每个主要的分析类型消耗的迭代次数,并当你调整一个或者多个仿真选项或命令时,用这个作为一个度量。

例如,表\ref{表6.3}展示了计算硬盘电路ch6-3.cir的直流偏置点需要的迭代次数和时间总量。对这个电路,增加.NODESET声明会把迭代次数从53减少到13,并把偏置点分析使用的时间减少56\%。试着给你的电路增加一个或者多个.NODESET声明,并在DCAN程序片段下查看列出的迭代次数。有了合适的设置,你经常可以把迭代次数减少50\%或者更多,并且消除相当一部分工作点计算需要的时间。

\begin{figure}[htbp]
\small
    \centering
    \includegraphics[width=0.5\textwidth]{figure/Chapter6/表6.3.png}
    \caption{偏置点运行时间数据}
    \label{表6.3}
\end{figure}

对于瞬态分析,观察NUMTTP,NUMRTP,和NUMNIT的值。NUMTTP是SPICE在瞬态仿真中使用的时间点数;这不是在输出文件中打印的点数。NUMRTP是倒转的时间点数或在给定的时间点收敛失败之后,SPICE切分时间步长并回退尝试重新求解解点的次数。NUMNIT是完成整个瞬态分析所需要的迭代次数。通过在.OPTIONS声明中提升ITL4的值,SPICE经常在瞬态分析中会把时间步长切分得更小。SPICE使用的反向时间点越少,仿真就会越快完成。对一个有效的仿真,NUMTTP,NUMRTP,和NUMNIT的值应该尽可能地小。通常,NUMTTP,NUMRTP,和NUMNIT的值会通过提高ITL4来减小。当在你的仿真中提升ITL4时,查看NUMTTP,NUMRTP,NUMNIT的值和瞬态仿真使用的时间总量。继续提升ITL4直到NUMTTP,NUMRTP,和NUMNIT无法获得进一步提高,仿真时间停止减少或者精度降低。

硬盘文件ch6-4.cir仿真了一个SR触发器的瞬态响应。在ch6-4.cir文件中,原始的电路描述用不同的ITL4值复制了几遍。表\ref{表6.4}展示了ITL4值,瞬态时间点数,反向时间点数,瞬态迭代次数,和整个的仿真时间。注意,把ITL4提高到40以上对仿真速度产生不了明显的提高。用RSPICE仿真这个电路来验证这些结果。

\begin{figure}[htbp]
\small
    \centering
    \includegraphics[width=0.5\textwidth]{figure/Chapter6/表6.4.png}
    \caption{瞬态运行时间数据}
    \label{表6.4}
\end{figure}

为了决定你的SPICE程序中容许的最大节点或者元件数,查看MAXMEM和MEMUSE值。SPICE使用一个长数据数组来存储电路描述,元件值,和所有的模型参数。同一个数据数组存储了所有的分析结果。数组的尺寸决定了给定程序容许的最大元件数或者最大分析点数。如果你的仿真存储需求超过可用的内存量,SPICE会在输出文件中打印最不友好的信息,并中止仿真。MAXMEM是你的SPICE版本能够解决的最大内存量。MEMUSE是在仿真中使用的内存量。数据数组必须同时保存电路的元件和所有分析数据。正因为此,当执行一个非常短的分析时,就说一个直流偏置点的计算,SPICE可以仿真数量最多的元件。相反,当仿真一个非常小的电路时,SPICE可以仿真最长的分析(大多数分析点)。

为了决定SPICE可以仿真的最大元件数,用几个元件创建一个电路网表。在电路网表中包含ACCT选项。仿真该电路,记录MAXMEM和MEMUSE值,并给原本的网表添加几个额外的元件。再一次仿真电路,记录新的MEMUSE值。两个MEMUSE值之间的区别代表了额外元件要求的内存量。每个元件的内存量和MAXMEM可以被用来决定SPICE可以仿真的理论最大元件数。硬盘文件ch6-5.cir包含了一个7阶和一个14阶的环状振荡器。当仿真的时候,MEMUSE值可以被用来计算电路中7个额外阶需要的内存量。表\ref{表6.5}展示了仿真每种电路需要的内存量。7个额外阶的增加需要4376个内存单元(或者每个反向器阶625个内存单元)。通过外插每阶的内存单元数到可用的200,000个内存单元数,用户会推断出在这个仿真中至多大概可有320阶(320个反相器)。

\begin{figure}[htbp]
\small
    \centering
    \includegraphics[width=0.5\textwidth]{figure/Chapter6/表6.5.png}
    \caption{运行数据或者最大元件数}
    \label{表6.5}
\end{figure}

很明显,当执行一个更复杂的分析的时候,最大元件数会被减少。

任何任务的最大仿真长度都可以做一个相似的计算。例如,为了决定是否执行一个长瞬态仿真,在电路上运行一个短瞬态仿真。之后在前次运行基础上增加20\%的瞬态时长。MEMUSE值之间的差别表示了在瞬态时长上增加20\%需要的内存量。最大瞬态时长可以通过较长瞬态运行使用的MAXMEM值和内存量计算得到。

硬盘文件ch6-6.cir包含了前面问题的同样的7阶环状振荡器。第一个电路文件仿真了4nS(带有100pS打印精度)瞬态行为。第二个仿真了8nS的行为(再一次带有100pS打印精度)。表\ref{表6.6}展示了两次运行使用的内存量。

\begin{figure}[htbp]
\small
    \centering
    \includegraphics[width=0.5\textwidth]{figure/Chapter6/表6.6.png}
    \caption{最大运行时间的运行数据}
    \label{表6.6}
\end{figure}

从瞬态持续时间的每纳秒内存量来看,用户会推断出这个电路同样的100pS打印精度会产生18$\mu$S的最大瞬态持续时间。

纯粹地讲,MAXMEM和MEMUSE都有4字节字的单位;所以,为了决定两者中的任意一种使用的真实内存字节数,用4乘以MAXMEM和MEMUSE值。结果就是SPICE正在使用的字节数。

\subsection{.OPTIONS列表}
LIST选项是一个标记,指导SPICE在输出中打印仿真的所有元件类型的列表。列表根据元件类型排序。所有的电阻列在最前,之后是所有的电容、电感,等等,直到电路中的所有元件被列完。包含在列表中的是每个元件的节点连接,元件值,任何可选元件参数,和子电路连接信息。图\ref{图6.2}是LIST选项的例子。

LIST选项最好用在拓扑相关错误的诊断中。错误的节点连接可以被快速发现并解决。该列表按照元件类型排序,这使得节点连接非常容易被读取。

\begin{figure}[htbp]
\small
    \centering
    \includegraphics[width=0.5\textwidth]{figure/Chapter6/图6.2.png}
    \caption{LIST选项输出}
    \label{图6.2}
\end{figure}

(原书166页和167页缺失。)

LIST产生电路网表中的部件列表,NODE产生电路中的节点和连接每个节点的元件列表。图\ref{图6.4}展示了输出文件中打印的节点连接信息。
\begin{figure}[htbp]
\small
    \centering
    \includegraphics[width=0.5\textwidth]{figure/Chapter6/图6.4.png}
    \caption{NODE选项输出}
    \label{图6.4}
\end{figure}

尽管NODE选项可以被用来调试拓扑相关问题,但是NODE不描述给定的部件的哪个端口连接到特定的节点。出于这个原因,LIST选项是一个相当好的拓扑调试工具。NODE选项的一个比较好的应用是辅助解决直流工作点不收敛。在直流工作点分析中,如果电路收敛失败,SPICE在输出中打印最后一次迭代的节点电压值列表。经常,这些电压值中的一个或者多个是极其不切实际地高和明显错误。当遇到一个或者多个这样的节点时,使用NODE选项产生连接到该节点的所有元件列表。一个或者多个这些元件正在产生不收敛问题。一旦问题元件被识别出来,使用在第\ref{chap:non-conver}章中提到的.NODESET或者OFF声明来纠正该不收敛问题。

\subsection{.OPTIONS OPTS}
OPTS选项是SPICE中最有用的选项之一。OPTS选项是一个标识,在SPICE仿真中指导SPICE打印每个可选参数的值。乍看起来,这个选项相当多余,因为用户必须在.OPTIONS行指明选项。但是OPTS选项打印所有用户选择的选项,不仅仅是在.OPTIONS声明上指明的那些。

OPTS选项中最好用的一个是决定所有SPICE选项的默认设置。决定任一SPICE仿真器的默认设置是极其重要的,因为不同的仿真器会赋不同的默认值,而这些值对仿真器的速度,精度,和收敛性质有着巨大影响。

为了决定SPICE中的默认设置,用一行除了OPTS选项外没有别的语句的.OPTIONS行来产生一个简单的网表。仿真结束后,输出文件会包含与图\ref{图6.5}所示的列表相似的章节。这些就是SPICE的默认选项设置。

\begin{figure}[htbp]
\small
    \centering
    \includegraphics[width=0.5\textwidth]{figure/Chapter6/图6.5.png}
    \caption{OPTS选项输出。}
    \label{图6.5}
\end{figure}

如果你在仿真中用一个或者多个设置在.OPTIONS行(除了OPTS声明)的选项来使用OPTS的话,声明的值,而不是默认的值,会打印在输出文件里。

\subsection{.OPTIONS LIMPTS=X}
LIMPTS=X是设置在输出文件中仿真或者打印的解点数量的上限的参数。LIMPTS是安装在SPICE中为了防止用户偶然消耗过量的计算机资源的几个安全网络中的一个。例如,如果下面行

.TRAN .1MS 20MS

被偶然输入为

.TRAN .1NS 20MS

SPICE会试着仿真200百万个数据点!

对今天的单用户计算机来说,过量的运行时间或者输出文件不是一个重大的问题。工作站或者个人电脑用户可以在没有重大回响的情况下轻易中止一个长仿真运行。但是当SPICE被开发的时候,使用了一个中央计算设备,批量任务,和行打印机输出。通常,直到打印输出被转运到输出房间,用户对他们任务是如何运行的一无所知。第二,.TRAN声明在输出文件中产生200百万个输出点。假定每页66行,一英寸厚的折叠式打印机纸有100页,那么200百万个点转换成打印机纸的厚度的话有将近半英里高!想想试着从打印机取走那一摞的计算机中心的人的表情!

默认地,LIMPTS在SPICE中被设置为201。许多供应商提供的SPICE版本把这个数重置为了一个更高的值。在读取输入文件的过程中,SPICE决定用户对仿真要求的点的个数。如果点数多于LIMPTS,SPICE在输出文件中打印错误信息并结束执行仿真。用户可以把该参数重置为他或她希望仿真的准确的点数,或者一个更大的默认值。(作者通常在大多数仿真中设置LIMPTS为10,000。)

\subsection{.OPTIONS NUMDGT=X}
输出选项的最后一个是NUMDGT=X参数。NUMDGT是电压和电流打印在输出文件中的有效位数。改变这个选项会调整SPICE使用的打印格式,不是仿真器的精度。默认地,NUMDGT被设置为4。这个参数容许的值是1到7。

\section{输出文件格式选项总结}
输出文件格式选项决定了打印在输出文件中的信息类型和格式。获取的信息会被用来调试网表相关的问题,最小化输出文件尺寸,决定默认的SPICE选项设置,和决定最大电路尺寸或者运行时间。

\section{数值控制选项}
数值控制选项包含四个子组。在第一个组中是收敛参数GMIN,RELTOL,ABSTOL,和VNTOL。第二个组包含迭代限制ITL1,ITL2,ITL3,ITL4,ITL5,和ILT6。瞬态分析选项组接着LVLTIM,METHOD,MAXORD,MU,TRTOL,和CHGTOL。最后,两个选项定义矩阵选主元组,PIVTOL和PIVREL。所有的数值控制选项都是参数化的选项,不是标记。参数化简单地说意味着选项的值必须遵循选项的名称。

\subsection{收敛性选项}
收敛性选项包含GMIN,RELTOL,ABSTOL,和VNTOL。第\ref{chap:non-conver}章描述了这些选项中的每一个,并讨论了他们如何影响电路的精度和收敛特性。

\subsubsection{.OPTIONS GMIN=X.}
参数GMIN=X代表了电路中任何元件容许的最小电导(最大电阻)。GMIN也是电路中与每个半导体PN结并联的小电导(大电阻)。GMIN可以等同为给定电路相关的寄生泄露。默认的GMIN=1e-12mhos。

GMIN最好的用途是辅助仿真器的收敛特性。对很多电路,从默认值提升GMIN会在不牺牲结果精度的前提下产生更好的整体收敛特性。为了给GMIN选择一个值,决定不影响正常电路操作的可以被放置在电路的任意两个节点之间的最小寄生电阻。设置GMIN=X为电阻值的倒数。

\begin{equation}
    GMIN=\frac{1}{(smallest parasitic resistance)}
\end{equation}

\subsubsection{.OPTIONS RELTOL=X.}
RELTOL=X参数是收敛要求的相对误差容限。为了求解电路方程,SPICE启动Newton-Raphson迭代解算法。Newton-Raphson算法一遍一遍地迭代,搜寻一套使得剩余电路方程遵循Kirchoff电压和电流定律的节点电压和支路电流。因为Newton-Raphson算法从来不真正“知道”合适的电路电压和电流,所以停止迭代的信号必须来自算法之外。Newton-Raphson算法的一个有用的性质是当算法接近准确解时,迭代之间的电压和电流值的变换会趋向0。

因为数字计算机在预测电压和电流之间的差值准确为0(因为舍入误差)时困难,所以SPICE假定当迭代电压和电流之间的差小于给定的误差容限的时候,Newton-Raphson算法就找到了准确值。误差容限的第一部分是迭代电压和电流之间的百分比变换。当电路中每个节点的迭代电压值之间的百分比变化小于RELTOL和迭代半导体支路电流之间的百分比变化小于RELTOL时,SPICE停止迭代过程。

RELTOL=X设置SPICE中容许的相对误差容限。RELTOL的默认值为.001,或者百分之一的十分之一。对大多数电路,这是速度和精度之间一个好的折中。更高的精度要求在给定解点上进行更多的迭代。更多的迭代转化成了更多的仿真时间。更低的精度要求更少的迭代,导致更快的仿真执行。Nagel证实每增加一位有效精度,求解需要的迭代次数会翻番\cite{chap6-2}。对大多数模拟电路,RELTOL的默认值产生满意的仿真精度和速度。对很多数字电路,为了更快的仿真执行,在对精度没有影响的前提下,RELTOL会被提升至.01或.05。对不同的电路实验RELTOL。采用一个能在速度和精度之间产生最好折中的值。RELTOL的值在参数VNTOL=X和ABSTOL=X被设定之前必须被确定。

\subsubsection{.OPTIONS VNTOL=X ABSTOL=X}
参数VNTOL=X和ABSTOL=X与RELTOL选项一起工作决定SPICE中迭代解算法的误差容限。对电路仿真中使用的大多数电压和电流值,除了一个例外,百分比变化标准工作地很好。当节点电压或者支路电流趋向于0时,百分比变化误差容限标准对迭代算法无法预测出一个好的终结点。正因为此,除了百分比变化标准,必须决定一个更低的绝对误差容限。当节点电压或者支路电流落入0,绝对误差容限会定义收敛。对电路中的节点电压,VNTOL=X时迭代之间的绝对误差容限,而ABSTOL=X是半导体支路电流的绝对误差容限。这两个选项参数决定了仿真器电压和电流的较低精度。

VNTOL的默认值是1$\mu$V。ABSTOL的默认值是1pA。SPICE的作者们设置这些给集成电路水平的电压和电流。对于任何非集成电路的电路(比如,板级,离散的,和特别地电源电路),参数VNTOL和ABSTOL应该被重置为与电路中看到的电压和电流水平对齐。

为了对给定电路选择VNTOL和ABSTOL的值,研究电路。确定电路中最低的电压幅度,并用RELTOL乘以这个值。结果用来设置VNTOL。对ABSTOL,确定电路中最低的电流幅度,并用RELTOL乘以这个值。结果用来设置ABSTOL。这两个方程对齐了SPICE的相对和绝对误差容限。用户应该注意,VNTOL和ABSTOL的值依赖于RELTOL的值。任何对RELTOL的修正都应该伴随着对ABSTOL和VNTOL的调整。

这里提醒用户,所有关于收敛参数的详细讨论都可以在第\ref{chap:non-conver}章找到。

\subsection{迭代限制选项}
接下来的一套选项是迭代限制选项。六个参数对不同的分析类型定义迭代限制。迭代限制全部从三个字母开始,ITL。

\subsubsection{.OPTIONS ITL1=X。}
ITL1=X参数定义计算直流工作点容许的迭代次数的上限。ITL1的默认值是100。经验上可以得到,60\%的电路能在100次内收敛,75\%的能在200次内收敛,92\%的能在500次内收敛。如果收敛需要的迭代次数超过500,那么电路中有一些东西可能被不正确地连接或者一个或多个节点需要用.NODESET或.IC声明初始化。

ITL1的最好使用是在提高直流工作点收敛特性上。对所有电路,把ITL1设置为500启动仿真。因为ITL1是被容许的迭代次数的上限,所以不要求过量的迭代来决定直流工作点的仿真不会被ITL1的提升影响。

SPICE在计算直流工作点,直流扫描,或者瞬态分析中使用的迭代次数可以通过使用本章早些时候讨论的ACCT选项被观察到。

\subsubsection{.OPTIONS ITL2=X。}
ITL2=X参数定义了在直流扫描分析中每步容许的迭代次数上限。ITL2的默认值是50。

ITL2的最好使用是在直流扫描分析中辅助收敛。对有着高增益的开关点电路,比如寄存器,比较器,运放,和触发器,设置ITL2为200。

\subsubsection{.OPTIONS ITL3=X。}
ITL3=X参数为迭代计数时间步长控制算法设置迭代次数的最低限制。当在瞬态仿真中使用迭代计数方法时,SPICE监测仿真中每个时间点上的迭代次数。如果迭代次数小于或者等于ITL3,时间步长控制算法在下一个时间点被计算前自动翻番步长尺寸。

ITL3的默认值是4次迭代,而且是凭经验决定的\cite{chap6-2}。无论Nagel还是作者都没有通过提高或者降低该值找到对迭代计数时间步长控制算法的任何明显的提高。对大多数电路,ITL3应该保持默认值。

\subsubsection{.OPTIONS ITL4=X。}
ITL4=X参数给计数迭代时间步长控制算法或者局部截断误差时间步长控制算法设置迭代上限。如果在给定的时间点,迭代次数超过ITL4,SPICE会丢掉当前的时间点,用因子8切掉时间步长,然后在新时间点重新尝试求解。ILT4的默认值是10次迭代。

ITL4的最好使用是增强收敛特性,增加瞬态仿真的速度。ITL4在瞬态仿真中有着巨大的影响。通过提高ITL4,在SPICE丢弃当前时间点之前,你提高容许的迭代次数。因为SPICE改变时间步长尺寸的不对称的方式,减少时间步长被切掉的次数可以大幅增加仿真速度。第\ref{chap:non-conver}章的瞬态不收敛小结解释了很多为什么提高ITL4会同时提高瞬态仿真的速度和收敛特性。

对瞬态仿真,设置ITL4为40。为了在瞬态仿真中提高速度和效率,使用NUMTTP,NUMRTP,和NUMNIT的值(看ACCT选项小节关于NUMTTP,NUMRTP,和NUMNIT的讨论),并相应调整ITL4。

\subsubsection{.OPTIONS ITL5=X。}
ITL5=X参数是另一张为了限制瞬态仿真使用的总迭代次数的置于SPICE中的安全网。在瞬态运行的每一个时间点上,SPICE在该时间点统计解要求的迭代次数。如果总数一旦超过ITL5,SPICE会停止瞬态仿真,打印仿真中直到那个点的结果,然后在输出文件中打印信息,宣称为了继续更长的仿真需要提高ITL5。ITL5的默认值为5000次迭代。

取决于电路,5000次迭代会要求只有很少一部分计算机时间。很多仿真在瞬态仿真中要求超过5000次迭代。用户有机会把ITL5提高到要求的近似迭代次数;简单地设置ITL5为0压制总的容许的迭代次数。

\subsubsection{.OPTIONS ITL6=X。}
参数ITL6=X同时是标志和参数值。ITL6的默认值是0。如果ITL6被设置为非零值,SPICE会丢弃正常的直流工作点计算,用电源步进算法取代之。宣示的ITL6的值作为电源步进算法中每一步的迭代限制。很多人错误地相信ITL6与过程中步数或者尺寸有关;其实不是这样。步的尺寸被固定了,无法被改变。ITL6只决定每步容许的迭代次数。

ITL6的最好的使用是在支持那些.NODESET声明无法使用的电路中的直流工作点的解。虽然使用.NODESET在寻找直流工作点时是一种比较推崇的方式,但是.NODESET无法被应用到子电路定义的节点。对很多大的电路,双稳态节点(比如寄存器)无法用.NODESET声明接触。在这些例子中,提高ITL6到400经常帮助实现收敛。

聪明地使用迭代限制是产生快速,准确,可收敛仿真的惊奇片段中的一个。惊奇的另一个部分是瞬态分析选项。这些会在下一节中被谈到。

\subsection{瞬态分析选项}
对大多数设计人员,瞬态分析是SPICE中最常用的类型。瞬态分析也是最复杂的分析模式,因此也是SPICE中最容易出错的分析类型。为了产生精确的瞬态仿真,用户必须理解瞬态分析算法的限制,该算法包含数值积分,时间步长控制,和六个只能用在瞬态分析的SPICE选项。

\subsubsection{.OPTIONS LVLTIM=X。}
参数LVLTIM=X定义瞬态仿真中SPICE使用的哪个时间步长控制算法。默认地,LVLTIM设置为2,调用局部截断误差(LTE)时间步长控制算法。尽管稍微快些,但是LTE时间步长控制算法比迭代计数方法更容易犯错。不给最大内部时间步长施加限制,LTE方法会在仿真异步,正弦,或者电导性电路时产生糟糕的结果。对这些电路,设置LVLTIM为1选择迭代计数时间步长控制算法。迭代计数时间步长控制算法自动限制内部时间步长的尺寸,并且在这些类型的电路上会更加可靠。关于LVLTIM=X和时间步长控制算法的详细讨论,看本书的第\ref{chap:5}章。

\subsubsection{.OPTIONS METHOD='yyyy'。}
METHOD='YYYY'选项定义计算电容电流和电感电压的数值积分方法。有效的设置是METHOD='GEAR'或METHOD='TRAP'。默认的设置是TRAP,不用惊奇,该设置选择梯形数值积分方法。梯形数值积分方法是一种相对快速,准确的方法,但梯形积分不是万无一失的。梯形方法围绕正确的解会有烦人的振荡趋向,特别在开关电路或者长的瞬态仿真中。虽然Gear方法积分没有振荡,而且在长的瞬态仿真中趋向于更稳定(保持靠近精确解),但代价是需要更长的仿真运行时间。

\subsubsection{.OPTIONS MAXORD=X。}
MAXORD=X参数定义Gear多阶积分方法的最高阶。当用户指定METHOD=GEAR时,SPICE使用Gear第二阶方法。更高阶的Gear方法也是可行的,包括Gear的第三,四,五,和六阶方法。为了利用更高阶方法中的一种,选项MAXORD必须被设置为会使用的最高阶。MAXORD=3指示第三阶Gear方法,MAXORD=4指示第四阶Gear方法,MAXORD=5指示第五阶Gear方法,而且最后,MAXORD=6指示第六阶Gear方法。

MAXORD=X参数有一个2的默认值,而且可以被设定在2和6之间。

应该注意的是,在理论上,更高阶的Gear方法应该在每一个时间点上产生更少的误差,而且会带来更少的总时间点和更快的瞬态仿真;由于高阶Gear方法牵扯的额外负担,然而,没有一个高阶方法明显比Gear第二阶方法快\cite{chap6-2}。

对于数值积分方法和每种电路类型使用的积分类型的更详尽的讨论,读者可以直接翻到本书的第\ref{chap:4}章。

\subsubsection{.OPTIONS MU=X。}
MU=X参数在SPICE2G.6用户指南\cite{chap6-1}中从没有被记录。像其他几个选项,MU同时作为标志和参数值。MU的默认值为5。

为了理解MU,梯形积分的方程和后向欧拉数值积分应该被比较。在这两种方法中存在惊人的相似性。如果有人仔细用一个单,被很好放置的变量书写方程,变量的值从梯形积分到后向欧拉积分会被用来改变方程。在SPICE中,MU就是那个变量。通过设置MU为5,默认设置,方程简化为梯形积分方法。通过设置MU为0,方程简化为后向欧拉积分方法,而通过把MU设置在0和5之间的某个地方,这两种方法的混合版就会被使用。读者可以转到第\ref{chap:4}章寻求关于积分方法和什么时候使用后向欧拉或者梯形积分的更详尽的讨论。

\subsubsection{.OPTIONS TRTOL=X CHGTOL=X。}
TRTOL=X和CHGTOL=X参数是更容易误解的参数中的两个。因为名字以TOL结束,更像RELTOL,ABSTOL,和VNTOL,很多人相信这些也是误差容限。不幸的是这个并不是那种情况。TRTOL和CHGTOL在局部截断误差(LTE)时间步长控制算法的开发过程中同时被安装,而且两个都只与LTE算法相关。

在LTE算法开发过程中,局部截断误差的预测值被发现比准确的截断误差值大7倍多\cite{chap6-2}。为了补偿这一偏离,叫作TRTOL的参数为了截断误差被加入方程,而且,不用惊奇,TRTOL的默认值为7。

虽然TRTOL在瞬态分析使用的全局步长尺寸上有着直接的关系,但是增加TRTOL增加步长尺寸,而减小TRTOL减小步长尺寸。因为TRTOL被加入是为了补偿截断误差的糟糕估计,所以这个参数最好被留在默认值7。

同时,可以观察到在特定的电路上,局部截断误差时间步长控制算法会通过产生无限小的步长尺寸“锁住”。这个小的步长尺寸发生在电容之上的电荷或者流过电感的磁通量比程序的误差容限(RELTOL,ABSTOL,和VNTOL)还小的时候。为了阻止这种情况发生,叫作CHGTOL的参数作为对电容电荷或者电感磁通量的限制被加入方程。无论何时电路产生最大的电容电荷或者电感磁通量小于CHGTOL,SPICE在LTE方程中使用CHGTOL的值以预测下一个时间步长。CHGTOL是一种阻止LTE时间步长控制算法失败的简单手段。

CHGTOL的默认值为1e-14库伦。降低这个值不会明显增加精度,只会增加延长整个瞬态仿真时间的可能性。出于这些原因,CHGTOL最好被留在默认值。

\subsection{数值主元选项}
最后两个数值控制选项也是最后两个加入SPICE的选项。程序的最后一个重大改变是加入了修正的数值选主元算法。选主元会重排矩阵,减少求解矩阵要求的操作次数。

\subsubsection{.OPTIONS PIVTOL=X PIVREL=X。}
PIVTOL=X和PIVREL=X参数都与SPICE中的数值选主元算法相关。PIVTOL参数定义了最小的被认为可接受的矩阵元素的数值值(矩阵元素)。如果元素小于PIVTOL,数值溢出条件(比如除0)就可以发生。为了阻止这种情况,SPICE检查每一个矩阵的元素以确保没有主元元素小于PIVTOL。PIVREL=X参数定义在给定的电导数组的列中的最大元素和PIVTOL之间的比例。

数值选主元是一种实验性很强的科学。对不同的电路,最优的主元是不同的。PIVTOL的默认值是1e-13,PIVREL的默认值是.001,而这两个都是根据64位双精度电导值经验决定的。出于这个原因,PIVTOL和PIVREL都应该被留在它们默认的值。

\subsection{数值控制选项总结}
为了实现快速,准确,可收敛的仿真,SPICE用户必须理解和知道如何设置18个形成数值控制选项组的参数。有质量的仿真要求有意义的用户输入。部分的输入是以这18个数值控制选项参数的形式支持的。本书的第\ref{chap:non-conver},\ref{chap:4},和\ref{chap:5}章详细描述了对给定电路每个参数应该如何被设置。学习这些章节,并对数值控制选项变得熟悉。当与电路仿真的其他方面比较的时候,在这18个参数上的时间投资会在有质量的仿真上产生最大的回报。

\section{MOS几何选项}
接下来的四个参数形成MOS几何选项。这四个都只能应用到MOS晶体管上。MOS晶体管门的几何对器件的I-V和C-V特性有着很大的影响。因为,在任何集成电路上,很多晶体管共享相同的几何形状,所以为了减少在电路网表中描述晶体管几何形状的精力,给SPICE加入了四个选项。

\subsection{.OPTIONS DEFL=X DEFW=X}
这些参数中的前两个是DEFL=X和DEFW=X参数。DEFL选项描述了SPICE在计算晶体管漏电流中使用的默认栅宽。在很多IC制造工厂,最小栅宽是通过制造工艺定义的,而设计人员修改栅宽可以调整晶体管的驱动电流。如果DEFL=X参数被设置为容许的最小栅宽,设计人员只需要指明元件线上的晶体管栅宽度。如果设计人员选择使用非默认的带栅宽的晶体管,简单给元件线上增加L=X的描述覆盖原本用DEFL声明的默认值。

栅宽参数DEFW=X定义了晶体管栅的默认宽度。与DEFL=X参数一样,DEFW=X通过在元件线上指明W=X会被覆盖。对DEFL=X和DEFW=X两者,长度和宽度的单位都是米。如果你用微米描述这些量,记着添加U(微米)规模的因子。DEFL和DEFW参数的默认值都是100微米。

\subsection{.OPTIONS DEFAD=X DEFAS=X}
就像栅长和宽会用默认设置描述一样,漏和源的面积会被设置为默认值。如果CJ模型参数被声明,漏和源的面积会被用在漏和源电容电容值的计算中。DEFAD=X定义了漏扩散区的默认值,而DEFAS=X定义了源扩散区的默认值。像长度和宽度参数一样,漏面积和源面积都是以平方米为单位的。

\section{其他选项}
最后四个选项落入了叫作其他选项的类别。

\subsection{.OPTIONS TNOM=X}
这些参数中的第一个是TNOM=X参数。TNOM定义了分析的正常温度。TNOM的值应该被用摄氏度表示。默认地,TNOM设置为27$^{\circ}$C。虽然这个对室温而言是热的,但假设是,即使在室温,半导体结也会产生热。TNOM代表这些器件结的温度。

在SPICE中所有温度相关的项会根据温度偏离TNOM而改变。TNOM应该是温度相关模型参数或者元件值提取的基础。

\subsection{.OPTIONS CPTIME=X LIMTIM=X}
CPTIME=X和LIMTIM=X参数是另外两个当初SPICE原始开发的时候就安装的选项,被加入是为了限制对于给定仿真任务可以使用的计算机资源的量。

CPTIME=X参数定义最大的可以用来仿真电路的CPU时间。因为计算机资源耗费的持续降低,大多数SPICE版本通过把这个默认值设置的相当高已经讨论过这个选项。默认地,CPTIME被设置为10亿秒。(对于你们中的那些正在寻找计算机看到底可以转换成多少天的人,让我给你们节省些时间。10亿秒大约是32年!)

像CPTIME,LIMTIM是另一个为了节省计算机资源的安全网。LIMTIM=X参数定义了产生输出容许的最大时间。这个参数在很多SPICE版本中已经被讨论过。默认地,LIMTIM被设置为2秒。

\subsection{.OPTION LVLCOD=X}
最后一个选项叫作LVLCOD。LVLCOD=X参数在SPICE早期的CDC版本中使用过。LVLCOD的默认值为0。如果LVLCOD被设置为1,SPICE的早期版本会调用一个叫作CODGEN的子函数。

在迭代的过程中,在每一次新的迭代中,SPICE必须求解系统方程阵列(在线性方程系列上执行高斯消去的计算机等效)。SPICE用几个级联的fortran do-loop求解那个系列。因为fortran do-loop在大的,稀疏的矩阵上趋向于相对低效,所以SPICE的作者决定增加一个可以绕过fortran do-loop并能用直接的机器指令取代解的函数。做这件事情的函数叫作CODGEN。CODGEN写机器指令直接求解系统方程。虽然CODGEN的加入可以减少10-30\%的仿真时间,但是该函数被从最新的SPICE版本移除了。CODGEN被从SPICE移除有两个原因:第一,因为编译器有效性和计算机速度都有了巨大的提高,而第二,因为机器指令是专门针对CDC计算机写成的。这使得移植SPICE到IBM或者Cray或者Vax或者任何其他计算机上都成了噩梦。出于这些原因,CODGEN函数已经被从大多数的SPICE版本中移除。

\section{对选项设置的一些建议}
对大多数的电路,性能会通过重置默认选项值中的几个得到彻底提升。下面是对所有电路都建议的设置。

.OPTIONS ACCT LIST ITL1=500 ITL2=200 ITL4=40 ITL5=0 LIMPTS=10K

其他需要被设置的选项包括GMIN=X,RELTOL=X,ABSTOL=X,VNTOL=X,LVLTIM=X,和METHOD='ABCD',但是所有这些必须被设置为适合于你的电路的值。第\ref{chap:non-conver},\ref{chap:4}和\ref{chap:5}章讨论了对不同类型的电路设置特定的选项。

\section{总结}
SPICE中的34个选项定义了输出文件中的打印信息,程序中使用的数值算法,以及程序的迭代限制。学会使用和知道对于给定的电路设置合适的选项定义了有经验的SPICE用户和菜鸟之间的区别,也定义了快速,准确,可收敛,高质量仿真结果和垃圾之间的区别。SPICE不像很多人相信的那样自动或者可靠。为了产生精确的结果,SPICE仿真要求有意义的用户输入。SPICE要求的输入包含精确网表的连续性,精确器件模型参数设置和元件值,以及精确的选项参数设置。


\bibliographystyle{plain}
\bibliography{Reference/chapter6-reference}