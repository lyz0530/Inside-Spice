\chapter*{前言}
\markboth{Preface}{前言}

尽管SPICE在1971年就出现了,但因为个人计算机的原因, SPICE的使用在过去10年里出现了爆发式增长。IBM对台式个人计算机的推广和SPICE的普遍可获得性(包括学术形式和商业形式)使得模拟电路仿真对几乎每一个电路设计人员都成为了可能。

作为一个设计者,作者多少年来一直在和SPICE斗争,因为尽管这个程序对辅助电路设计有着无穷的潜力,但程序的实际使用却受限于收敛失败、步长控制错误和数值积分失败。这些不同种类的问题要么使程序终止,要么会在仿真输出上引入错误。因为这些失败,仿真结果要不很难获取要不包含着大量的不准确。

尽管这些错误几乎会规律性发生,但几乎没有什么关于纠正这些仿真问题的东西写下来过。本文的动机来源于现在的设计人员缺乏关于如何克服这些仿真障碍的信息。获得精确的高质量的仿真结果要求学会如何克服这些仿真失效。

也许有一天,数学家和程序员有能力开发出一个总能收敛到正确解和总能产生精确结果的电路仿真器。在那一天到来前,作为SPICE的用户,我们必须学会如何克服这些限制。

\section{本书组织架构}
这本书可以同时作为教程或者参考资料。

作为教程,文中包括了对$RSPICE^{TM}$电路仿真器和$RGRAPH^{TM}$图像后处理器的有限复制。第3、4、5和6章包含的电路例子应该用RSPICE/RGRAPH来仿真。这些电路集成在了文中,表明了SPICE常见的失效机制和用来纠正这些失效应该采取的措施。本书附带的硬盘里包含了所有的文中提到的电路文件,是以章节来组织的。

作为参考资料,书中包含了原理图,逐步纠正常见仿真器失效的步骤。本书同时包含了多个仿真失效的例子。这些例子可以被用来识别读者仿真结果中相似的失效情况。

\section{RSPICE/RGRAPH硬盘}
附带的RSPICE/RGRAPH硬盘包括了RSPICE电路仿真器和RGRAPH图形后处理器。这块硬盘同时包含了本书中用到的电路例子。安装程序会自动处理生成合适的目录结构,复制文件和指定合适的目录指针这些工作。

\section{要求}
安装这些程序和文件需要2MB的硬盘资源。安装程序会生成如下的目录和子目录。

$\backslash$RCGV33

$\backslash$RCGV33$\backslash$RSPICE

$\backslash$RCGV33$\backslash$CH2

$\backslash$RCGV33$\backslash$CH3

$\backslash$RCGV33$\backslash$CH4

$\backslash$RCGV33$\backslash$CH5

$\backslash$RCGV33$\backslash$CH6

$\backslash$RCGV33$\backslash$DEMO

RSPICE和RGRAPH文件随后会被复制到合适的目录下。

\section{安装}
我们会假定你正在通过软盘来安装该软件。要安装这些程序,简单地输入:

X:intall X: Y:

这里X是软盘指定的驱动程序,Y是硬盘指定的驱动程序。安装程序会完成剩余的安装事务。

\section{章节内容总结}
第1章是对电路仿真的简要回顾。对CANCER, SPICE1, SPICE2和SPICE3的发展进行历史性回顾会被包含在这一章中。本章接着会阐述电路仿真在设计流程中如何可以成为一种辅助手段和有成果有效地对电路仿真需要些什么。

第2章会阐述SPICE如何工作,展示SPICE通过输入文件的元件构造系统方程,SPICE如何迭代出解,SPICE如何计算DC偏执,DC扫描,AC扫频和瞬态分析这些解。理解SPICE的基础算法对理解如何纠正常见仿真器的失效机制是非常重要的。

第3章检查不收敛问题。不收敛是仿真用户最常见和最令人沮丧的问题之一。但是大多数的不收敛问题可以通过简单使用SPICE中的选项和控制来解决。第3章追寻不收敛的起因,并且给出原理上的,逐步的步骤来消除(大多数)不收敛仿真失效。

第4章是重点提高瞬态分析精度的两章内容的头一章。在第4章,SPICE的三种积分方法会被检验。数值积分失效会把错误引入仿真结果,但SPICE没有警告用户仿真失败的函数。作为仿真用户,我们必须学会从仿真结果中监测仿真失效并且采取合适的纠正动作。

第5章通过展示SPICE中的步长控制算法来结束瞬态分析上的讨论。像数值积分一样,SPICE不能够监测到步长控制失效。用户必须学会在仿真输出中找寻步长失效。第5章检测了不同的步长控制失效,并且说明了合适的纠正性措施。

第6章检测了SPICE中的34个选项。每一个选项都给出了详尽的解释和建议设置。

附录A比较了几个受欢迎的代理商提供的电路仿真器之间默认选项设置。

程序SPICE在本书中被参考了很多次。在这本书中,SPICE特指Berkeley SPICE2G.6。